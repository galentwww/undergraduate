\section{施工组织设计}

\subsection{施工流向、程序及顺序}
\subsubsection{施工流向}
\subsubsection{施工程序}
\subsubsection{施工顺序}

\subsection{施工组织机构及主要管理人员职能}
\subsubsection{施工组织机构}
\subsubsection{主要管理人员职责}

\subsection{施工总平面布置说明}
\subsubsection{现场道路}
\subsubsection{现场材料堆放}
\subsubsection{现场垂直运输系统}
\subsubsection{现场用电布置}
\subsubsection{现场临时设施}

\subsection{施工总进度计划及工期保证措施}
\subsubsection{整体工期控制目标}
\subsubsection{主要施工程序进度计划控制}
\subsubsection{工期保证措施}

为确保本工程如期、保质保量地完成,项目部将在以下几个方面采取相应的措施:\\

(1) 施工组织保障措施\\

为保证施工计划完成,我们将选派有丰富的现场施工组织管理经验的、并曾担任过类似工程的项目经理担任该工程项目经理。
以加强项目部管理能力和组织协调能力,及时解决施工中遇到的问题,保证施工中各环节、各专业、各工种之间的协调与平衡。
对工程质量、安全进行监督及人力、财力、物力的统一调度,确保施工顺利进行,杜绝因管理不善造成施工脱节、资源浪费、工期延误现象的发生。\\

(2) 合理的施工方案\\

\ding{192} 充分熟悉本工程的设计图纸,对拟定的施工组织设计、施工方案及方法进行认真的分析比较,作到统筹组织、全面安排,确保总体目标计划。在施工过程中制定阶段性工期控制点,确保按期完工。针对工程特点,采用分段流水施工方法,减少技术间歇,突出重点,制定严密的、紧凑的、合理的施工穿插,尽可能压缩工期,加快施工进度。

\ding{193} 合理地加入投入机械,提高机械化作业程度,充分满足工程所需的人、财、物要求。

\ding{194} 对各班组进行教育,责任到人。各区分段划分负责人,举行施工质量、进度、安全等评比,奖优罚劣。

\ding{195} 引进先进竞争机制,在整个工程过程中,每月评比质量、进度、安全最差班组及最优班组。当月最差班组下月的工程量相应少一半,多出一半部分由最优班组承接。\\


(3) 搞好各种资源的供应\\

\ding{192} 根据施工组织设计的要求和施工进度计划中各个阶段控制点的要求,编制劳动力进场计划、材料进场计划、机械设备进场计划、资金使用计划,以保证各种资源能满足施工需要。

\ding{193} 物资材料计划有明确的材料数量、规格和进场时间,现场材料储备应有一定的库存量,以保证工程进度提前或节假日运输困难时工程对物资的需要,确保现场施工正常进行。

\ding{194} 劳动力进场要保证质量,工人进场前必须进行严格的培训和考核。保证足够数量的劳动力。

\ding{195} 按照计划施工机械进场前对机械进行必要的维护、保养和试运转工作,保证所有机械进场后能够投入正常使用。使用期间加强维护与保养,确保机械能正常运行。

\ding{196} 保证足够的资金用于施工,专款专用,确保本工程正常运转。\\


(4) 严格的管理与控制\\

\ding{192} 强化项目方法管理,推行项目方法施工,实行项目经理负责制,设立能协调各方面关系的调度指挥机构,配备素质高、能力强,有开拓精神的管理班子,确保施工进度。

\ding{193} 全面推行计划动态管理,控制工程进度,建立主要形象进度控制点,运用网络计划跟踪技术和动态管理方法,做到周保旬,旬保
月,坚持月平横,周调度、工期倒排,确保总进度的计划实施。

\ding{194} 认真做好施工中的计划统筹、协助与控制。严格坚持落实每周工地施工例会制度,做好每日工程进度安排,确保各项计划落实。
编排详细的工程施工总进度计划,并采用微机管理技术,对施工计划实行动态管理;建立主要工程进度控制点,
围绕总进度计划,编制月、周施工进度计划,作到各分部分项工程的实际进度按计划要求进行;每期根据前期完成情况和其他预测情况变化,
对当期计划和后期计划、总计划进行重新调整和部署,确保按原定或因非施工原因调整了的期限交工。

\ding{195} 实行奖励机制,拟定拿出一定的资金作为目标管理和科技进步奖励基金,充分调动全体施工人员的积极性和创造性,力保各项目标按期实现。

\ding{196} 制定各工序的操作规程和质量标准,强化施工现场管理,做到安全文明施工,努力实现施工管理的标准化、科学化、合理化,使施工有条不紊。

\ding{197} 强化项目部内部管理人员效率与协调,增强与业主的联系,加强对劳务人员的控制和与各供货商的协作,并确保各方及个人的职责分工,
减少扯皮现象,争取将围绕本工程建设的各方面人员充分调动起来,共同完成工期总目标。

\ding{198} 创造和保持施工现场各方面各专业之间的良好的人际关系,使现场各方认清其间的相互依赖和相互制约的关系。特别是加强同交通疏导、
材料运输、周围居民的协调,增进与业主、监理、设计单位的联系和配合,及时解决问题。\\

\subsection{主要项目施工方法和技术措施}
\subsubsection{土方开挖工程}
\subsubsection{土方回填工程}
\subsubsection{钢筋工程}
\subsubsection{模板及支撑工程}
\subsubsection{混凝土工程}
\subsubsection{脚手架工程}
\subsubsection{砌体工程}
