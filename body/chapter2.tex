\section{施工组织设计}

\subsection{施工流向、程序及顺序}
\subsubsection{施工流向}
\subsubsection{施工程序}
\subsubsection{施工顺序}

\subsection{施工组织机构及主要管理人员职能}
\subsubsection{施工组织机构}
\subsubsection{主要管理人员职责}

(1) 项目经理岗位职责\\

\ding{192} 全面负责本工程的一切事务,认真贯彻执行《建筑法》、《合同法》和国家有关劳动保护法令和制度以及公司的各项管理制度,
贯彻安全第一、预防为主的方针,按规定搞好安全防范措施,把安全工作落到实处,在各种经济承包中必须包括安全生产、做到讲效益必须讲安全,抓生产首先必须抓安全。

\ding{193} 认真熟悉施工图纸、组织编制施工组织设计方案和施工安全技术措施,组织编制工程总进度计划表和月进度计划表及各施工班
组的月进度计划表。会同项目部相关人员精选强有力的施工队伍,编制工程进度计划及人力、物力计划和机具、用具、设备计划,做到合理组织施工,如发现计划工期无法保证应及时进行调整。

\ding{194} 制定适合本工程项目的管理细则、方案及措施,组织项目部例会,合理安排、科学引导、顺利完成本工程的各项施工任务。
对项目质量全面负责,负责制定质量考核目标,并监督检查。

\ding{195} 对项目成本控制负责,安排、搞好项目的成本核算(按单项和分部分项)单独及时核算,并将核算结果及时通知公司各部门部的管理人员,
以便及时改进施工计划及方案,争创更高效益。及时向各班组下达施工任务书及材料限额领料单。

\ding{196} 对现场文明施工管理负责,组织制定现场文明各项施工管理规定。根据本工程施工现场情况合理规划布局现场平面图,安排、实施、创建文明工地。要求布局合理、经济。

\ding{197} 深入实际了解员工的生活工作学习情况,采纳员工中的合理化建议,妥善解决好施工中出现的各类问题,保质保量顺利完成本工程施工任务。\\

(2) 施工员岗位职责\\

\ding{192} 在项目经理的直接领导下开展工作,贯彻安全第一、预防为主的方针,按规定搞好安全防范措施,把安全工作落到实处,做到讲效益必须讲安全,抓生产首先必须抓安全。

\ding{193} 认真熟悉施工图纸、编制各项施工组织设计方案和施工安全、质量、技术方案,编制各单项工程进度计划及人力、物力计划和机具、用具、设备计划。并监督计划执行情况。

\ding{194} 组织新进场员工技术培训,并做好每道工序的技术交底工作。

\ding{195} 编制文明工地实施方案,根据本工程施工现场合理规划布局现场平面图,安排、实施、创建文明工地。

\ding{196} 编制工程总进度计划表和月进度计划表及各施工班组的月进度计划表,并跟踪实施情况。

\ding{197} 向各班组下达施工任务书及材料限额领料单。配合项目经理工作。\\

(3) 安全员岗位职责\\

\ding{192} 在项目经理领导下,全面负责监督实施施工组织设计中的安全措施、并负责向作业班组进行安全技术交底。

\ding{193} 检查施工现场安全防护、用电安全、机械设施、等是否符合安全规定和标准。如发现施工现场有不安全隐患,应及时提出改进措施,
督促实施并对改进后的设施进行检查验收。对不改进的,提出处置意见报项目负责人处理。

\ding{194} 正确填报施工现场安全措施检查情况的安全生产报告,定期提出安全生产的情况分析报告的意见。

\ding{195} 处理一般性的安全事故。

\ding{196} 按照规定进行工伤事故的登记,统计和分析工作。

\ding{197} 同各施工班组及个人签订安全生产协议书。

\ding{198} 随时对施工现场进行安全监督、检查、指导,并做好安全检查记录。对不符合安全规范施工的班组及个人进行安全教育、处罚,并及时责令整改。

\ding{199} 在安全检查工作中不深入、不细致及存在问题不提出意见又不向上级汇报,所造成的责任事故,应承担全部责任及后果。\\

(4) 质检员岗位职责\\

\ding{192} 在项目经理领导下,负责检查监督施工组织设计的质量保证措施的实施,组织建立各级质量监督保证体系。

\ding{193} 严格监督进场材料的质量、型号、规格、监督各项施工班组操作是否符合规程。

\ding{194} 按照规范规定的分部分项检验方法和验收评定标准,正确进行自检和实测实量,填报各项检查表格,
对不符合工程质量标准、质量要求返工的分部分项工程,写出返工意见并出具罚款单。保证按规定对每一层次、工序不漏检,
并配合施工进度不影响施工。凡经检验不合格的工程,应及时报告项目经理以便采取措施。

\ding{195} 提出工程质量通病的防治措施,提出制订新工艺、新技术的质量保证措施建议。

\ding{196} 对工程的质量事故进行分析,提出处理意见。

\ding{197} 向每个施工班组做(质量验收评定标准)交底。

\ding{198} 每个房间施工都应在施工段(墙、柱、梁、板)贴上质量检查验收表。只有经检验合格才能进入下一工序施工。 \\

(5) 项目经理安全生产岗位责任制\\

\ding{192} 工程项目经理对工程项目的安全生产负有全面责任

\ding{193} 建立和健全项目体安全管理网络,确定安全管理目标,组织编制安全保证计划

\ding{194} 根据工程特点,加强对分包单位的控制和管理

\ding{195} 认真执行各项安全生产规章制度,明确项目体各管理岗位的安全生产职责,负责检查项目体的安全生产责任制落实情况

\ding{196} 适时组织对工程项目部的安全体系评审和协调

\ding{197} 落实安全保证计划的资源配置

\ding{198} 依据施工组织设计,落实各项安全技术措施,对分部分项工程必须派人进行安全技术交底

\ding{199} 落实专人负责检查特种作业人员持证情况,对新的分包单位进入工地,要有针对地进行安全教育,制止违章操作

\ding{200} 发生工伤事故成立即时组织抢救,保护现场,迅速如实上报,并参加事故调查处理,按“四不放过”的原则,落实各项整改工作

\ding{201} 自觉接受上级安全生产部门的监督和管理\\


(6) 安全员安全生产岗位责任制\\

\ding{192} 贯彻执行劳动保护法规、制度,做好安全生产的宣传,教育和管理工作

\ding{193} 贯彻安全保证计划中的各项安全技术措施,组织参与安全设施、施工用电、施工机械的验收

\ding{194} 对进入现场使用的各种安全用品及机械设备进行各项必要材料的存档及抽查工作

\ding{195} 对分包单位进行安全、消防检查,指导或协助生产负责人解决生产中不安全问题

\ding{196} 参加组织安全、消防活动和安全、消防检查,制止违章作业。对事故隐患开具整改单限期整改并复验,
对有严重险情的有权决定轶作业并及时上报。对违反劳动保护、安全生产法规的行为,经说服劝阻无效时,有权越级上报或举报

\ding{197} 督促有关分包单位做好对新工人、特殊工种人员以及换岗人员的安全技术教育,做好个人安全档案工作

\ding{198} 对施工组织设计中的安全技术措施在实施过程中进行督促检查

\ding{199} 负责对分包单位安全生产的责任考核工作

\ding{200} 参加伤亡事故调查、分析、处理工作。按照“四不放过”的原则,做好安全事故的统计、分析和报告以及归档工作\\

(7) 项目部生产班组长岗位责任制\\

\ding{192} 按照施工方案,组织劳动力进场,彻底做好班组的施工工艺和安全技术措施交底工作

\ding{193} 监督、检查本班组操作工人按图纸、规范、施工方案施工

\ding{194} 组织班组进行自检、互检和交接检工作,发现不合格项目及时组织工人进行整改,确保本班组工作面的质量符合标准

\ding{195} 负责传达项目部的各项管理内容和上报班组各项情况,及时进行调解

\ding{196} 认真遵守安全规章和有关安全生产制度,对本组人员在生产中的安全健康负责

\ding{197} 搞好安全活动日,开好班前、班后安全会,对新调入的工人进行现场班组级安全教育

\ding{198} 组织本班级职工学习施工技术和安全规程及制度,检查执行情况,在任何情况下,均不得违章,不得擅自动用机械、电气、架子等设备

\ding{199} 经常检查施工现场的安全生产情况,加强安全自检,发现问题及时解决,不能解决的采取措施并及时上报

\ding{200} 发生工伤事故要详细记录并及时上报,组织全组人员认真分析,提出防范措施。发生重大伤亡事故要保护现场并立即上报项目部主管\\

\subsection{施工总平面布置说明}
\subsubsection{现场道路}
\subsubsection{现场材料堆放}
\subsubsection{现场垂直运输系统}
\subsubsection{现场用电布置}
\subsubsection{现场临时设施}

\subsection{施工总进度计划及工期保证措施}
\subsubsection{整体工期控制目标}
\subsubsection{主要施工程序进度计划控制}
\subsubsection{工期保证措施}

为确保本工程如期、保质保量地完成,项目部将在以下几个方面采取相应的措施:\\

(1) 施工组织保障措施\\

为保证施工计划完成,我们将选派有丰富的现场施工组织管理经验的、并曾担任过类似工程的项目经理担任该工程项目经理。
以加强项目部管理能力和组织协调能力,及时解决施工中遇到的问题,保证施工中各环节、各专业、各工种之间的协调与平衡。
对工程质量、安全进行监督及人力、财力、物力的统一调度,确保施工顺利进行,杜绝因管理不善造成施工脱节、资源浪费、工期延误现象的发生。\\

(2) 合理的施工方案\\

\ding{192} 充分熟悉本工程的设计图纸,对拟定的施工组织设计、施工方案及方法进行认真的分析比较,作到统筹组织、全面安排,确保总体目标计划。在施工过程中制定阶段性工期控制点,确保按期完工。针对工程特点,采用分段流水施工方法,减少技术间歇,突出重点,制定严密的、紧凑的、合理的施工穿插,尽可能压缩工期,加快施工进度。

\ding{193} 合理地加入投入机械,提高机械化作业程度,充分满足工程所需的人、财、物要求。

\ding{194} 对各班组进行教育,责任到人。各区分段划分负责人,举行施工质量、进度、安全等评比,奖优罚劣。

\ding{195} 引进先进竞争机制,在整个工程过程中,每月评比质量、进度、安全最差班组及最优班组。当月最差班组下月的工程量相应少一半,多出一半部分由最优班组承接。\\


(3) 做好各种资源的供应\\

\ding{192} 根据施工组织设计的要求和施工进度计划中各个阶段控制点的要求,编制劳动力进场计划、材料进场计划、机械设备进场计划、资金使用计划,以保证各种资源能满足施工需要。

\ding{193} 物资材料计划有明确的材料数量、规格和进场时间,现场材料储备应有一定的库存量,以保证工程进度提前或节假日运输困难时工程对物资的需要,确保现场施工正常进行。

\ding{194} 劳动力进场要保证质量,工人进场前必须进行严格的培训和考核。保证足够数量的劳动力。

\ding{195} 按照计划施工机械进场前对机械进行必要的维护、保养和试运转工作,保证所有机械进场后能够投入正常使用。使用期间加强维护与保养,确保机械能正常运行。

\ding{196} 保证足够的资金用于施工,专款专用,确保本工程正常运转。\\


(4) 严格的管理与控制\\

\ding{192} 强化项目方法管理,推行项目方法施工,实行项目经理负责制,设立能协调各方面关系的调度指挥机构,配备素质高、能力强,有开拓精神的管理班子,确保施工进度。

\ding{193} 全面推行计划动态管理,控制工程进度,建立主要形象进度控制点,运用网络计划跟踪技术和动态管理方法,做到周保旬,旬保
月,坚持月平横,周调度、工期倒排,确保总进度的计划实施。

\ding{194} 认真做好施工中的计划统筹、协助与控制。严格坚持落实每周工地施工例会制度,做好每日工程进度安排,确保各项计划落实。
编排详细的工程施工总进度计划,并采用微机管理技术,对施工计划实行动态管理;建立主要工程进度控制点,
围绕总进度计划,编制月、周施工进度计划,作到各分部分项工程的实际进度按计划要求进行;每期根据前期完成情况和其他预测情况变化,
对当期计划和后期计划、总计划进行重新调整和部署,确保按原定或因非施工原因调整了的期限交工。

\ding{195} 实行奖励机制,拟定拿出一定的资金作为目标管理和科技进步奖励基金,充分调动全体施工人员的积极性和创造性,力保各项目标按期实现。

\ding{196} 制定各工序的操作规程和质量标准,强化施工现场管理,做到安全文明施工,努力实现施工管理的标准化、科学化、合理化,使施工有条不紊。

\ding{197} 强化项目部内部管理人员效率与协调,增强与业主的联系,加强对劳务人员的控制和与各供货商的协作,并确保各方及个人的职责分工,
减少扯皮现象,争取将围绕本工程建设的各方面人员充分调动起来,共同完成工期总目标。

\ding{198} 创造和保持施工现场各方面各专业之间的良好的人际关系,使现场各方认清其间的相互依赖和相互制约的关系。特别是加强同交通疏导、
材料运输、周围居民的协调,增进与业主、监理、设计单位的联系和配合,及时解决问题。\\

\subsection{主要项目施工方法和技术措施}
\subsubsection{土方开挖工程}
\subsubsection{土方回填工程}
\subsubsection{钢筋工程}
\subsubsection{模板及支撑工程}
\subsubsection{混凝土工程}
\subsubsection{脚手架工程}
\subsubsection{砌体工程}
