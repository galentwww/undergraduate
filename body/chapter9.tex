\section{经济技术性分析}

21 世纪是高科技时代,土木工程将会引进更多的高新技术,土木工程发展极为迅猛, 
其实践和研究已取得显著成就。而随着社会的不断发展及环境不断恶化,
人们也逐渐加强了对环境保护的重视,在追求土木工程舒适性,实用性,
美观性的同时也逐渐加强了对其环保功能的重视,因此,想要确保土木工程的长期稳定的发展,
就要合理的将可持续发展观融入到其中,有效的实现新时期土木工程的可持续发展。
在这个竞争日益激烈的市场环境下,经济技术效益是建筑企业生存的保障,
而建筑企业的经济技术效益与建筑经济技术有着必然的联系。众所周知,建筑是一大能耗行业,
所需要投入的人力、物力、财力非常大,如果建筑企业不能做好经济技术,势必就会影响到建筑企业的投入成本,
进而影响到经济技术效益。另外,成本管理作为建筑工程管理一项重要的工作,能够反映出建筑企业的施工质量,
如果企业不能做好经济技术,势必就会影响到建筑工程质量。 

\subsection{节约能源消耗}

建筑节能是近年来对建筑工程提出的一种全新的设计理念,
也是当前建筑工程领域广泛实践的新技术。土木工程在施工中需要消耗大量的能源,
已驱动机械设备运转,和施工人员的正常生活。如果不将能源消耗进行严格控制,
将对我国能源产生战略性的影响。在土木工程中提倡绿色施工的概念,
对于缓解我国能源危机,提高环境保护力度,提高人们的生活质量有重要意义。 

\subsection{提高施工水平}

传统的土木工程技术虽然已经可以实现大规模建设工程的建设任务,
但是在完成工程的同时也带来了极大的环境问题,通过提倡绿色施工技术,
在施工建设过程中对环境保护和能源消耗提出严格的限制要求,
可以促使施工单位不断改进技术,寻求更好的施工方案,促进土木工程行业的迅速发展。

\subsection{技术装备管理水平的影响}
先进的技术和装备是企业在建筑行业竞争中体现优势的主要因素,
先进的管理水平更能体现设备和技术的价值最大程度提高生产效率从而降低造价成本。