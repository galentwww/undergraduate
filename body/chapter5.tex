\section{模板工程专项安全施工方案}
\subsection{编制依据}

(1) 《建筑工程施工质量验收统一标准》(GB50300-2001)

(2) 《建筑结构荷载规范》(GB50009-2001)

(3) 《建筑施工模板安全技术规范》(JGJ162-2008)

(4) 《建筑施工扣件式钢管脚手架安全技术规范》(JGJ130-2011)

(5) 《建筑施工脚手架安全技术统一标准》(GB51210-2016)

(6) 《建筑施工安全检查标准》(JGJ59-2011)

(7) 《混凝土结构工程施工及验收规范》(GB50204-2015) 

\subsection{模板支撑架搭设要求}

(1) 必须设置纵横扫地杆。纵向扫地杆应采用直角扣件固定在距底座上皮不大于200mm 处的立杆上,
横向扫地杆也应采用直角扣件固定在紧靠纵向扫地杆下方的立杆上。当立杆基础不在同一高度上时,必须将高出的纵向扫地杆向低处延长两跨与立杆固定,高低差不应大于 1m。

(2) 立杆应采用对接接头,且接头位置不应设置在同一步内,同一步立杆的两个相隔接头在高度方向错开的距离不宜小于 500,
各接头中心至柱节点的距离不宜大于步距的 1/3。

(3) 纵向水平杆接长宜采用对接扣件连接,对接扣件应交错布置,两根相邻纵向水平杆接头不宜设置在同步或同跨内,
不同步或不同跨两个相邻接头在水平方向错开的距离不应小于 500m,各接头中心至最近主节点的距离不宜大于纵距的 1/3。

(4) 搭接长度不应小于 1m,应等距离设置 3 个旋转扣件固定,端部扣件盖板边缘至搭接纵向水平杆端的距离不应小于 100mm。

\subsection{模板计算书}
\subsubsection{基本参数}

(1) 钢筋混凝土板厚 $120mm$,板的模板采用木胶合板厚 $15mm$,面板下次愣采用 $40\times 80mm$ 木方,间距 $300mm$,主楞采用 $60\times 100mm$ ,木方间距 $500mm$,支撑体系采用 $\phi 48.3\times 3.6mm$ 钢管。

(2) 梁尺寸 $300\times 700mm$,模板采用木胶合板厚 $18mm$,侧模次楞采用 $40\times 80mm$ 木方, 布置间距 $200mm$,采用 $M20$ 对拉螺栓加固,
间距为 $250mm$,梁底次楞采用 $40\times 80mm$ 木方,次楞间距 $150mm$,梁模板主楞采用 $40\times 80mm$ 木方,间距 $500mm$,支撑体系采用 $Φ48.3\times 3.6mm$ 钢管。

(3) 柱尺寸 $500\times 500mm$ ,计算高度取 $9.0m$,模板采用木胶合板厚 $18mm$,次楞采用 $40\times 80mm$ 木方,间距为 $115mm$,
柱箍间距 $400mm$,最下一层柱箍距柱根间距为 $200mm$,最上一层柱箍距柱顶间距为 $200mm$,柱箍螺栓采用 $M18$。


\subsubsection{模板安全性验算}

(1) 对板的安全性验算\\

\quan{1} 楼板验算\\

根据《建筑结构荷载规范》(GB50009-2012),查得相关构件的标准荷载值如下:

模板自重标准值取 $G_{1k}=0.5 kN/m^2$

混凝土自重标准值取 $G_{2k}=24\times 0.12=2.88 kN/m^2 $

钢筋自重标准值取 $G_{3k}=1.1\times 0.12=0.132 kN/m^2$

施工活荷载标准值取 $2.5kN$($2.5kN/m$)

当活荷载作为均布线性荷载控制时:

\[q_1=0.9\times 1.0\times [1.2(500+2880+132)+1.4\times 2500]=6942 N/m\]

当恒荷载作为均布线性荷载控制时:

\[q_1=0.9\times 1.0\times [1.35(500+2880+132)+0.7\times 1.4\times 2500]=6471 N/m\]

$q$ 取较大值,故 $q=6942 N/m$

当作为集中荷载控制时:

\begin{align*}
    q_2&=0.9\times 1.0\times 1.2\times (500+2880+132)=3792 N/m\\
    P&=0.9\times 1.0\times 1.4\times 2500=3150N
\end{align*}

施工荷载为均布线荷载时的弯矩值为:

\begin{align}
    \label{fx:5.0}
    M_1=0.1ql^2=0.1\times 6.942\times 0.32=0.06kN\cdot m
\end{align}

施工荷载为集中荷载时的弯矩值为:

\begin{align}
    M_2&=0.1ql^2+0.213pl\\
    &=0.1\times 3.79\times 0.32+0.213\times 3.15\times 0.3 \notag\\
    &=0.23 kN \cdot m \notag
\end{align}

弯矩取最大值计算,故 $M=0.23kN\cdot m$\\

对板做强度验算,按照三跨连续梁计算,

\begin{align}
    \label{fx:5.1}
    \sigma =\frac{M}{W}
\end{align}

式中,$M$ 为模板受到最大弯矩值;$W$ 为截面模量,矩形的截面模量公式为:

\begin{align}
    \label{fx:5.2}
    W=bh^2/6
\end{align}

式中,$b$ 为模板计算宽度;$h$ 为模板厚度。代入数据可得:

\begin{align*}
    W&=1000\times 152/6=37500 mm^3\\
    \sigma &=230000/37500=6.1 N/mm2<f=22 N/mm^2
\end{align*}

故板的强度满足要求。\\

对板做挠度验算,按照三跨连续梁计算,

\[q=0.9\times 1.0\times (500+2880+132)=3002 N/m\]

\begin{align}
    \label{fx:5.3}
    V_{max}=\frac{0.990ql^4}{100EI}
\end{align}

$E$ 为模板的弹性模量,取 $E=10000 N/mm^2$;$I$ 为截面惯性矩,计算公式为:

\begin{align}
    \label{fx:5.4}
    I=\frac{bh^3}{12}
\end{align}

代入数据得 $I=1000\times 15^3/12=281250 mm^4$。将截面惯性矩代入公式 \ref{fx:5.3} 得

\begin{align*}
    V&=\frac{0.99\times (3.002\times 10^2\times 300\times 10^2)}{100\times 10000\times 281250}\\
    &=0.058 mm < [v]=1/400=0.75 mm
\end{align*}

故板的挠度满足要求。\\

\quan{2} 板的次楞验算\\

板次楞采用 40×80mm 木方,间距 300mm,主楞间距 500mm。次楞的荷载按三跨连续梁计算受力,对板的次楞做强度验算:

\begin{align*}
    q_1&=0.9\times [1.2(500+2880+132)+1.4\times 2500]\times 0.3=1972 N/m \\
    q_2&=0.9\times (500+2880+132)\times 0.3=1.00 N/m
\end{align*}

分别按照公式 \ref{fx:5.0}、\ref{fx:5.2} 计算得出最大弯矩值与截面模量:

\begin{align*}
    M&=0.1\times 1.972\times 0.5^2=0.047 kN \cdot m\\
    W&=40\times 80^2 /6=42667 mm^2
\end{align*}

随后按照公式 \ref{fx:5.1} 求出板的次楞的强度:

\[
    \sigma = \frac{47000}{42667}=1.112 N/mm^2< f=17N/mm^2
\]

故次楞的强度满足要求。\\

对板的次楞做挠度验算,根据公式 \ref{fx:5.4} 算出次楞的最大截面惯性矩 $I$

\[
    I=40\times 80^3 /12=1706667 mm^4
\]

再将结果代回公式 \ref{fx:5.3},便可求出次楞的挠度设计值:

\begin{align*}
    V&=\frac{0.99\times 1 \times 500^4}{100\times 10000\times 1706667}\\
    &=0.0204 mm<[v]=1/400=1.25mm
\end{align*}

故次楞的挠度也满足要求。\\

\quan{3} 板的主楞验算\\

主楞采用 $60\times 100mm$ 木方,间距 $500mm$,板支撑在主楞上间距 $900mm$。由于木楞自重较小,因此计算时将其忽略
,主楞按照三跨梁计算,对主楞做强度验算:

\begin{align*}
    q_1&=0.9\times [1.2(500+2880+132)+1.4\times 2500]\times 0.5=3781 N/m \\
    q_2&=0.9\times (500+2880+132)\times 0.5=1.58 N/m
\end{align*}

按照主楞的最大弯矩公式求出主楞的最大弯矩值:

\begin{align}
    M=0.289q_1l=0.289\times 3.78\times 0.9=0.98 kN \cdot m
\end{align}

按照公式\ref{fx:5.2} 计算得出主楞的截面模量:

\[
    W=60\times 100^2 /6=100000 mm^2
\]

随后按照公式 \ref{fx:5.1} 求出板的主楞的强度:

\[
    \sigma = \frac{980000}{100000}=9.8 N/mm^2< f=17N/mm^2
\]

故主楞的强度满足要求。\\

对板的主楞做挠度验算,根据公式 \ref{fx:5.4} 算出次楞的最大截面惯性矩 $I$

\[
    I=60\times 100^3 /12=5000000 mm^4
\]

再将结果代回公式 \ref{fx:5.3},便可求出次楞的挠度设计值:

\begin{align*}
    V&=\frac{0.99\times 1580 \times 900^3}{100\times 10000\times 5000000}\\
    &=0.23 mm<[v]=1/400=2.25mm
\end{align*}

故主楞的挠度也满足要求。\\

\quan{4} 板的立杆验算\\

楼板支模高度 $9.0m$,属于高支模,立杆采用 $\phi 48.3\times 3.6$ 钢管,底部设置一道拉结杆,向上每 $1.5m$ 设置一道拉结杆。

查规范可得各个关键部件的自重标准值:

支撑自重为 $0.1444\times 9.0=1.29kN$

混凝土自重为 $24\times 0.12\times 0.5\times 0.9=1.29kN $

钢筋自重为 $1.1\times 0.12\times 0.5\times 0.9=0.05kN$

按照各个部件的自重标准值,能够求出恒荷载与活荷载的值:

恒荷载为 $1.27+1.29+0.05=2.63kN$

活荷载为 $2.5\times 0.5\times 0.9=1.125kN$

当活荷载控制时:

\[N_1=0.9\times (1.2\times 2.63+1.4\times 1.125)=4.25kN\]

当恒荷载控制时:

\[N_2=0.9\times (1.35\times 2.63+1.4\times 0.7\times 1.125=4.18kN\]

立杆轴向力取值取最大荷载值,即 $N=4.25 kN$

计算满堂脚手架的立杆长度的公式如下:

\begin{align}
\label{fx:5.5}
l_0&=k\mu_1(h+2a)\\
\label{fx:5.6}
l_0&=k\mu_2h
\end{align}

式中:
$k$ 为满堂支撑架立杆计算长度附加系数,取 1.185;   

$h$ 为步距 1.5m;                           

$a$ 为立杆伸出顶层水平杆中心线至支撑点长度 20mm;

$\mu_1$、$\mu_2$ 为满堂支撑架整体稳定因素的单杆计算长度系数,
根据《建筑施工扣件式脚手架安全技术规范》取值为 $\mu_1=1.54$、$\mu_2=1.951$

将上述数据代入公式 \ref{fx:5.5}、\ref{fx:5.6} 得:

\begin{align*}
    l_1&=1.185\times 1.54\times (1.5+2\times 0.2)=3.467 m\\
    l_2&=1.185\times 1.951\times 1.5=3.468 m
\end{align*}

$l_0$ 取二者最大值,即 $l_0=3.468 m$,随后根据公式 $\lambda=l_0/i$ 求出长细比为

\[
    \lambda = \frac{3.468}{1.59}=218
\]

查表可得轴心受压构件的稳定系数 $\phi =0.153$,再将上述所有数据代入下式可得出杆的弯曲正应力 $\sigma $

\begin{align}
\sigma &=\frac{N}{\phi A}\\
&=\frac{4180}{0.153\times 506} \notag\\
&=53.9 N/mm^2<f=205 N/mm^2 \notag
\end{align}

故主杆满足设计要求。\\

(2) 对梁的安全性验算\\



(3) 对柱的安全性验算\\



\subsection{模板安装及拆除}
\subsubsection{模板安装}

\subsubsection{模板拆除}

(1) 模板拆除应该经过相关管理人员的批准,按照国家标准《混凝土结构工程施工质量验收规范》(GB50204)的有关规定执行;

(2) 当混凝土未达规定强度,或已经达到规定强度,但需提前拆模的,必须经过计算和主管部门的批准后方可拆除;

(3) 大体积混凝土的拆模时间除了要满足混凝土强度要求之外,还应使得混凝土内外温差降低到 25 摄氏度以下时方可拆模;

(4) 拆模前应确定所使用的工具有效可靠,扳手等工具必须挂在工具袋内;

(5) 拆模的顺序和方法应该按照模板的设计规定执行,当无设计要求时,应按照先支的后拆、后支的先拆、先拆非承重模板
后拆承重模板,并应从上至下进行拆除;

(6) 多人同时作业时,应分工明确,统一信号,应预留出足够的操作面,人员要站在安全处;

(7) 在拆除互相搭连并影响其他后拆模板的支撑时,应布设临时支撑。拆模时逐块拆卸,不得使用大锤和撬棍成片撬落或砸倒拉倒;

(8) 拆除有洞口的模板时,应采取防止操作人员坠落的措施,洞口模板拆除后,应按国家先行标准《建筑施工高处作业安全技术规范》(JGJ80)
的有关规定进行防护。

\subsection{模板工程安全措施}

(1) 模板施工属高空作业,作业人员必须佩戴安全帽,安全绳并设置妥当,经医生检查认为不适宜高空作业的人员,不得进行高空作业

(2) 工作前应检查使用的工具是否牢固可用,扳手等工具必须用绳系在身上,钉子必须放在工具袋内;

(3) 安装与拆除五米以上的模板时,应搭设脚手架并设置防护栏杆,严谨上下共同作业;

(4) 遇六级以上的大风时应停止高空作业,雨雪后应先清扫施工现场,等待场地不滑时在进行作业;

(5) 两人抬运模板时要互相配合,协同工作。传递模板和工具时应先用绳子系牢固后再升降,不得随意乱抛。钢模板装拆时上下应有人接应,钢模板及其配件应随装随拆随运;

(6) 不得在脚手架上堆放模板;

(7) 支撑不得搭设在门窗框和脚手架上,斜撑和拉杆应设置在 1.8m 以上

(8) 支模过程中如遇中途停止,应将支撑,搭头,柱头等固定妥善,拆模间歇时应将已活动的模板和支撑等运走或妥善堆放,防止踏空;

(9) 模板上的预留洞口,应在留设后盖好;混凝土板上的预留洞口应在模板拆除之后盖好。

\subsection{成品保护}

(1) 对违反模板安全操作规范的,有损害模板安全使用现象的行为应及时制止个纠正,多层板拆除后应及时清理并分规格摆放,码放的高度不得大于 1.5 米;

(2) 吊装物体时,应轻吊轻放,不准碰撞已施工模板;

(3) 不经相关人员同意,任何人不得私自拆模,拆除模板以不损坏墙体,表面及棱角为准;

(4) 安装和拆除模板时不得用大锤砸多层板,以免使多层板翘曲变形以及砸坏硂成品;

(5) 多层板运输和堆放时,应做好防水工作,堆放场地也要有排水措施。

