\section{模板工程专项安全施工方案}
\subsection{编制依据}

(1) 《建筑工程施工质量验收统一标准》(GB50300-2001)

(2) 《建筑结构荷载规范》(GB50009-2001)

(3) 《建筑施工模板安全技术规范》(JGJ162-2008)

(4) 《建筑施工扣件式钢管脚手架安全技术规范》(JGJ130-2011)

(5) 《建筑施工脚手架安全技术统一标准》(GB51210-2016)

(6) 《建筑施工安全检查标准》(JGJ59-2011)

(7) 《混凝土结构工程施工及验收规范》(GB50204-2015) 

\subsection{模板支撑架搭设要求}

\subsection{模板计算书}
\subsubsection{基本参数}

\subsubsection{模板安全性验算}

\subsection{模板安装及拆除}
\subsubsection{模板安装施工工艺}

\subsubsection{模板拆除}

\subsection{模板工程安全措施}

\subsection{成品保护}

(1) 对违反模板安全操作规范的,有损害模板安全使用现象的行为应及时制止个纠正,多层板拆除后应及时清理并分规格摆放,码放的高度不得大于 1.5 米;

(2) 吊装物体时,应轻吊轻放,不准碰撞已施工模板;

(3) 不经相关人员同意,任何人不得私自拆模,拆除模板以不损坏墙体,表面及棱角为准;

(4) 安装和拆除模板时不得用大锤砸多层板,以免使多层板翘曲变形以及砸坏硂成品;

(5) 多层板运输和堆放时,应做好防水工作,堆放场地也要有排水措施。

