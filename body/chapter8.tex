\section{施工现场事故应急预案}
\subsection{总则}
\subsubsection{编制目的}

\subsubsection{编制依据}

\subsubsection{适用范围}

\subsection{应急组织体系}
\subsubsection{事故应急救援组织体系主要职责}

\subsection{应急响应}
\subsubsection{响应分级}

\subsubsection{响应程序}

\subsubsection{应急资源调配}

\subsubsection{应急救援}

\subsubsection{扩大应急}

\subsection{事故应急处置}
\subsubsection{事故应急处置基本要求}

\subsubsection{二级响应处置措施}

\subsubsection{一级响应处置措施}

\subsubsection{事故应急处置结束}

\subsection{事故后期处置}

\subsection{事故应急保障措施}
\subsubsection{应急队伍保障}

\subsubsection{应急物资装备保障}

\subsubsection{其他保障}

\subsection{应急预案管理}
\subsubsection{应急预案修订与备案}

\subsection{火灾事故应急演练}
\subsubsection{指导思想}

\subsubsection{组织与安排}

\subsubsection{火灾应急演练过程}

\subsubsection{演练总结}

\subsection{施工现场专项应急预案}
\subsubsection{高处坠落事故专项应急救援预案}

\subsubsection{机械伤害事故专项应急救援预案}

\subsubsection{火灾事故专项应急救援预案}

\subsection{应急预案的评审与改进}
