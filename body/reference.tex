\begin{center}
    \section*{\zihao{-2}  \textbf{参考文献}}
    \end{center}
\addcontentsline{toc}{section}{参考文献}

{\zihao{5} \songti 

[1]	中华人民共和国住房和城乡建设部. 《建筑桩基技术规范》[J]. 岩土力学, 2008(11): 3020–3020.

[2]	张文友, 徐华荣, 黄红政, 等. 对《建筑施工扣件式钢管脚手架安全技术规范》JGJ130—2011中满堂支撑架计算的解析[J]. 施工技术, 2012, 41(17): 78-80+83.

[3]	陈云钢, 郭正兴. 高大支模施工安全风险管理的探讨及对《建筑施工模板安全技术规范》的理解[J]. 四川建筑科学研究, 2012, 38(02): 305–310.

[4]	谭亚学. 《工作危害性分析》在事故分析中的应用[J]. 安全、健康和环境, 2004(12): 27-29+35.

[5]	晓刚 谢, 胡忠日, 梅秀娟, 等. 火灾风险评估方法及工程实例应用[J]. 消防科学与技术, 2009, 28(01): 29–32.

[6]	张明轩, 朱月娇, 翟玉杰, 等. 建筑工程高处坠落事故的故障树分析方法研究[J]. 煤炭工程, 2008(02): 112–114.

[7]	方甫兵. 建筑火灾风险评估方法应用研究[D]. : 昆明理工大学, 2008.

[8]	山东省建设厅. 建筑基坑工程检测技术规范(GB50497-2009)[M]. : 中国计划出版社, 2009.

[9]	编写建设部建筑管理司组织. 建筑施工安全检查标准[M]. : 同济大学出版社, 2012.

[10]	杜荣军. 脚手架结构的设计规定和计算方法[J]. 建筑技术, 1999, 30(8): 532–535.

[11]	刘群, 贾莉, 黄俊. 脚手架事故危险源辨识及预防[J]. 山西建筑, 2015, 41(12): 242–244.

[12]	张树义国振喜. 实用建筑结构静力计算手册(精)[M]. : 机械工业出版社, 2009.

[13]	事故树分析在排桩基坑工程安全评价中的应用研究--《岩土工程学报》2011年06期[EB/OL][2020-03-26]. http://www.cnki.com.cn/Article/CJFDTotal-YTGC201106026.htm.

[14]	ZHANG S, SULANKIVI K, KIVINIEMI M, 等. BIM-based fall hazard identification and prevention in construction safety planning[J]. Safety Science, 2015, 72: 31–45.

[15]	MCMAHON A. Book reviews - Carolyn Postgate, David Oates \& Joan Oates. The excavation at Tell al-Rimah; the pottery (Iraq Archaeological Reports 4). 
1997. 275 pages, 101 plates. London: British School of Archaeology in Iraq; 0-85668-700-6 paperback £48[J]. Antiquity, 1999, 73(282): 957–958.

[16]	MA L. Construction of intelligent building sky-eye system based on multi-camera and speech recognition[J]. 
International Journal of Speech Technology, 2020, 23(1): 23–30.

[17]	本社. GB50009-2001建筑结构荷载规范[M]. : 中国建筑工业出版社, 2003.


}