%===============  封面  =================
\smallskip
\begin{center}

\vspace*{1.2cm}
{\linespread{1.25}\selectfont
\heiti{\zihao{-1} \textbf{沈阳建筑大学本科毕业设计(论文)}} \\
\vspace*{2.2cm}
{\zihao{2} 河南省旅游中心项目安全施工组织设计 }\\
\heiti{\zihao{3} \textbf{Safety Construction Organization design of\\ Henan Tourism Center} }\\}
\vspace*{3.5cm}

\zhongsong
\begin{tabular}{cc}
 \zihao{-3} 学\ \ \ \ \ \ \ \ 院:&\underline{\makebox[7cm][c]{\zihao{-2}土木工程学院}} \\ 
 \\
 \zihao{-3} 专\ \ \ \ \ \ \ \ 业: & \underline{\makebox[7cm][c]{\zihao{-2}安全工程}} \\ 
 \\
 \zihao{-3} 学生姓名: & \underline{\makebox[7cm][c]{\zihao{-2}曲俊宇}} \\ 
 \\
 \zihao{-3} 学\ \ \ \ \ \ \ \ 号: & \underline{\makebox[7cm][c]{\zihao{-2}1602120210}} \\ 
 \\
 \zihao{-3} 指导教师: & \underline{\makebox[7cm][c]{\zihao{-2}刘家喜}} \\ 
 \\
 \zihao{-3} 评阅教师: & \underline{\makebox[7cm][c]{\zihao{-2}}} \\ 
 \\
 \zihao{-3} 完成日期: & \underline{\makebox[7cm][c]{\zihao{-2}}} \\ 
 \\
\end{tabular} 

\vspace*{2.2cm}
\xingkai{\zihao{-2} 沈阳建筑大学} \\
\heiti{\zihao{-4} Shenyang Jianzhu University }\\
\thispagestyle{empty}
\end{center}
\clearpage
%=====================原创性声明===========
\begin{center}
{\zihao{2} \textbf{学位论文原创性声明}}
\end{center}

\songti \zihao{-3} \linespread{1.25} \selectfont {本人郑重声明:本人所呈交的毕业设计(论文),是在指导老师的指导下独立进行研究所取得的成果。
毕业设计(论文)中凡引用他人已经发表或未发表的成果、数据、观点等,均已明确注明出处。
除文中已经注明引用的内容外,不包含任何其他个人或集体已经发表或撰写过的科研成果。
对本文的研究成果做出重要贡献的个人和集体,均已在文中以明确方式标明。\\}

本声明的法律责任由本人承担。\\

\begin{flushleft}
\zihao{4} 作者签名: \quad\quad\quad\quad  \quad\quad\quad\quad \quad\quad\quad\quad   \quad\quad\quad\quad 日期:\quad\quad 年 \quad  月  \quad  日\\
\end{flushleft}
\clearpage
\begin{center}
\heiti{\zihao{2} \textbf{关于使用授权的声明}}
\end{center}

\songti \zihao{-3} \linespread{1.25} \selectfont {本人在指导老师指导下所完成的毕业设计(论文)及相关资料(包括图纸、试验记录、原始数据、实物照片、图片、录音带、设计手稿等),
知识产权归属沈阳建筑大学。本人完全了解沈阳建筑大学有关保存、使用毕业设计(论文)的规定,
本人授权沈阳建筑大学可以将本毕业设计(论文)的全部或部分内容编入有关数据库进行检索,
可以采用任何复制手段保存和汇编本毕业设计(论文)。如果发表相关成果,一定征得指导教师同意,
且第一署名单位为沈阳建筑大学。本人离校后使用毕业毕业设计(论文)或与该论文直接相关的学术论文或成果时,
第一署名单位仍然为沈阳建筑大学。\\}
\begin{flushleft}
    \zihao{-3} 作者签名: \quad\quad\quad\quad  \quad\quad\quad\quad \quad\quad\quad\quad   \quad\quad\quad\quad 日期:\quad\quad 年 \quad  月  \quad  日\\
    \zihao{-3} 导师签名: \quad\quad\quad\quad  \quad\quad\quad\quad \quad\quad\quad\quad   \quad\quad\quad\quad 日期:\quad\quad 年 \quad  月  \quad  日\\ 
\end{flushleft}

\clearpage

%%=============设计(论文)任务书===========
%\begin{center}
%\zihao{-2}\textbf{\songti 本科生毕业设计(论文)任务书} 
%\end{center}
%\smallskip
%\renewcommand{\arraystretch}{1.3}
%\begin{tabular}{lll}
%\zihao{4} \textbf{\songti 学生姓名: 曹宇} & & \zihao{4} \textbf{\songti 专业班级:\quad\quad 船海1006班} \\ 
%\zihao{4} \textbf{\songti 指导教师:徐海祥}&\makebox [3cm] & \zihao{4} \textbf{\songti 工作单位:\quad 武汉理工大学} \\ 
%\end{tabular}\\
%\begin{tabular}{lll}
%\zihao{4} \textbf{\songti 设计(论文)题目:}& \zihao{4} \textbf{\songti  武汉理工本科论文\LaTeX 模板 } &\\ 
%\zihao{4} \textbf{\songti 设计(论文)主要内容:} \\
%\end{tabular} \\ 
%\begin{enumerate}
%\item \LaTeX 环境的配置
%\item 主要字体的控制和数学公式的选用
%\item 图表和代码的粘贴
%\end{enumerate}
%\begin{tabular}{ll}
%\zihao{4} \textbf{\songti 要求完成的主要任务:}
%\end{tabular} \\ 
%\begin{enumerate}
%\item 选择合适的\TeX 编辑系统
%\item 学习如何使用控制代码完成排版
%\item 合理的安排学习和科研的时间来发展自己兴趣爱好
%\end{enumerate}
%\begin{tabular}{ll}
%\zihao{4} \textbf{\songti 必读参考资料:}
%\end{tabular}
%\begin{enumerate}
%\item \LaTeX  \quad User Manual
%\item  字体设计的艺术
%\end{enumerate}
%\begin{tabular}{lll}
%\zihao{4} \textbf{\songti 指导教师签名: }&\makebox [4cm]& \zihao{4} \textbf{\songti 系主任签名:} \\
%& & \zihao{4} \textbf{\songti 院长签名(章)}
%\end{tabular}
%\thispagestyle{empty}
%\clearpage
%==========本科生毕业设计(论文)开题报告  =============
%\begin{center}
%\zihao{-2} \textbf{\songti 武汉理工大学}\\
%\zihao{-2} \textbf{\songti 本科生毕业设计(论文)开题报告} 
%\end{center}
%\begin{tabular}{|l|}
%\hline \rule[-2ex]{0pt}{5.5ex} \makebox[13.5cm][l]{\zihao{4} \heiti 1、目的及意义(含国内外的研究现状分析) } \\ 
%\quad \LaTeX 是国际通行的科技论文排版软件,国际上科研机构和大学都采用它写作\\
%\quad 国内著名高校都有自己的本科生\LaTeX 模板供毕业生使用\\
%\quad 但是武汉理工大学还没有本科生\LaTeX 模板可以参考\\
%\quad 人类的价值在于创造而不是索取 \\
%\hline \rule[-2ex]{0pt}{5.5ex}  \zihao{4} \heiti
%2、基本内容和技术方案\\ 
%\quad 采用GITHUB托管降低代码维护成本\\
%\quad 加入在线\TeX 编辑器的使用简介 \\
%\quad 授人以渔,注重方法和理念的引导\\
%\hline \rule[-2ex]{0pt}{5.5ex}  \zihao{4} \heiti
%3、进度安排 \\ 
%\quad 离 deadline 两个月吃喝玩乐 \\
%\quad 离 deadline 一个月吃喝玩乐 \\
%\quad 离 deadline 半个月吃喝玩乐 \\
%\quad 离 deadline 一个星期狂写论文 \\
%\hline \rule[-2ex]{0pt}{5.5ex} \zihao{4} \heiti
%4、指导教师意见 \\ 
%\quad 曹宇同学是个好同志\\
%\quad 曹宇同志是个好同学\\
%\quad 本表格是支持跨页的长表格,你可以复制上面的内容进行测试\\
%\quad 具体方法是将tabular改为 longtable然后再编译\\
%\makebox[10cm][r]指导教师签名:\\
%\makebox[12cm][r]\quad 年\quad 月\quad 日\\
%\hline 
%\end{tabular} 
%\thispagestyle{empty}
