%!TEX program = xelatex
\documentclass[zihao=-4,a4paper]{ctexart}
%==================== 数学符号公式 ============
\usepackage{xeCJK}
\newcommand{\zhongsong}{\CJKfontspec{STZhongsong}}%华文中宋,请自行下载字体并安装
\newcommand{\xiaosi}{\fontsize{12pt}{18pt}\selectfont}            % 小四, 1.5倍行距
\newcommand{\sihao}{\fontsize{14pt}{21pt}\selectfont}            % 四号, 1.5 倍行距
\newcommand{\xiaosan}{\fontsize{15pt}{22pt}\selectfont}        % 小三, 1.5倍行距
\usepackage{fontspec}
\setmainfont{Times New Roman}

\usepackage{amsmath}                 % AMS LaTeX宏包
\usepackage[ruled]{algorithm2e}              %伪代码
%\usepackage{amssymb}                % 用来排版漂亮的数学公式
%\usepackage{amsbsy}
\usepackage[style=1]{mdframed}
\usepackage{amsthm}
\usepackage{amsfonts}
\usepackage{mathrsfs}                % 英文花体字 体
\usepackage{bm}                      % 数学公式中的黑斜体
\usepackage{bbding,manfnt}           % 一些图标,如 \dbend
\usepackage{lettrine}                % 首字下沉,命令\lettrine
\usepackage{gbt7714}                 %配置gb7714引用格式

\def\attention{\lettrine[lines=2,lraise=0,nindent=0em]{\large\textdbend\hspace{1mm}}{}}
\usepackage{longtable}
\usepackage[toc,page]{appendix}
\usepackage{geometry}                % 页边距调整
\geometry{top=3.5cm,bottom=2.5cm,left=2.5cm,right=2.5cm}
%\usepackage{relsize}                % 调整公式字体大小:\mathsmaller,\mathlarger
%\usepackage{caption2}               % 浮动图形和表格标题样式
\usepackage{booktabs}                %三线表上下加粗
\usepackage{diagbox}                 % 分类表头
%%%使用带圈数字作为教主
\usepackage{pifont}
\usepackage[symbol*,stable]{footmisc}
\DefineFNsymbols{circled}{{\ding{192}}{\ding{193}}{\ding{194}}
{\ding{195}}{\ding{196}}{\ding{197}}{\ding{198}}{\ding{199}}{\ding{200}}{\ding{201}}}
\setfnsymbol{circled}               %带圈脚注
%====================公式按章编号==========================
\numberwithin{equation}{section}
\numberwithin{table}{section}
\numberwithin{figure}{section}
%================= 基本格式预置 ===========================
\usepackage{fancyhdr}
\pagestyle{fancy}
\fancyhf{}  
\fancyhead[C]{\zihao{5}  \songti 沈阳建筑大学本科毕业设计(论文)题目}
\fancyfoot[C]{~\zihao{5} \thepage~}
\renewcommand{\headrulewidth}{0.65pt} 
\ctexset{
    section = {
        format = \bfseries \xiaosan \heiti ,
        name = {, }
    },
    subsection = {
        nameformat = \bfseries \sihao \heiti
    },
    subsubsection = {
        nameformat = \bfseries \xiaosi \heiti
    }
}
%================== 图形支持宏包 =========================
\usepackage{subfigure}
\usepackage{graphicx}                % 嵌入png图像
\usepackage{color,xcolor}            % 支持彩色文本、底色、文本框等
\usepackage{hyperref}                % 交叉引用
\usepackage{caption}
\usepackage{multirow}                  %合并表格
% set up labelformat and labelsep for figure
\captionsetup{labelsep=quad}
\captionsetup{figurewithin=section}

\renewcommand{\thesubfigure}{(\arabic{subfigure})} %还可设置图编号显示格式,加括号或者不加括号

%==================== 源码和流程图 =====================
\usepackage{listings}                % 粘贴源代码
\usepackage{tikz}                    
\usepackage{tikz-3dplot}
\usetikzlibrary{shapes,arrows,positioning}
%===================   正文开始    ===================
\begin{document}
%===================  定理类环境定义 ===================
\newtheorem{example}{例}              % 整体编号
%\newtheorem{algorithm}{算法}
\newtheorem{theorem}{定理}            % 按 section 编号
\newtheorem{definition}{定义}
\newtheorem{axiom}{公理}
\newtheorem{property}{性质}
\newtheorem{proposition}{命题}
\newtheorem{lemma}{引理}
\newtheorem{corollary}{推论}
\newtheorem{remark}{注解}
\newtheorem{condition}{条件}
\newtheorem{conclusion}{结论}
\newtheorem{assumption}{假设}
%==================重定义 ===================
\renewcommand{\contentsname}{目 ~~ 录}     
\renewcommand{\abstractname}{摘 ~~ 要} 
\renewcommand{\refname}{参考文献}     
\renewcommand{\indexname}{索引}
\renewcommand{\figurename}{图}
\renewcommand{\tablename}{表}
\renewcommand{\appendixname}{附录}
\renewcommand{\proofname}{证明}
\renewcommand{\algorithmcfname}{算法} 
%\renewcommand{\algorithm}{算法} 
%============== 封皮和前言 =================
%===============  封面  =================
\smallskip
\begin{center}

\vspace*{1.2cm}
{\linespread{1.25}\selectfont
\heiti{\zihao{-1} \textbf{沈阳建筑大学本科毕业设计(论文)}} \\
\vspace*{2.2cm}
{\zihao{2} 河南省旅游中心项目安全施工组织设计 }\\
\heiti{\zihao{3} \textbf{Safety Construction Organization design of\\ Henan Tourism Center} }\\}
\vspace*{3.5cm}

\zhongsong
\begin{tabular}{cc}
 \zihao{-3} 学\ \ \ \ \ \ \ \ 院:&\underline{\makebox[7cm][c]{\zihao{-2}土木工程学院}} \\ 
 \\
 \zihao{-3} 专\ \ \ \ \ \ \ \ 业: & \underline{\makebox[7cm][c]{\zihao{-2}安全工程}} \\ 
 \\
 \zihao{-3} 学生姓名: & \underline{\makebox[7cm][c]{\zihao{-2}曲俊宇}} \\ 
 \\
 \zihao{-3} 学\ \ \ \ \ \ \ \ 号: & \underline{\makebox[7cm][c]{\zihao{-2}1602120210}} \\ 
 \\
 \zihao{-3} 指导教师: & \underline{\makebox[7cm][c]{\zihao{-2}刘家喜}} \\ 
 \\
 \zihao{-3} 评阅教师: & \underline{\makebox[7cm][c]{\zihao{-2}}} \\ 
 \\
 \zihao{-3} 完成日期: & \underline{\makebox[7cm][c]{\zihao{-2}}} \\ 
 \\
\end{tabular} 

\vspace*{2.2cm}
\xingkai{\zihao{-2} 沈阳建筑大学} \\
\heiti{\zihao{-4} Shenyang Jianzhu University }\\
\thispagestyle{empty}
\end{center}
\clearpage
%=====================原创性声明===========
\begin{center}
{\zihao{2} \textbf{学位论文原创性声明}}
\end{center}

\songti \zihao{-3} \linespread{1.25} \selectfont {本人郑重声明:本人所呈交的毕业设计(论文),是在指导老师的指导下独立进行研究所取得的成果。
毕业设计(论文)中凡引用他人已经发表或未发表的成果、数据、观点等,均已明确注明出处。
除文中已经注明引用的内容外,不包含任何其他个人或集体已经发表或撰写过的科研成果。
对本文的研究成果做出重要贡献的个人和集体,均已在文中以明确方式标明。\\}

本声明的法律责任由本人承担。\\

\begin{flushleft}
\zihao{4} 作者签名: \quad\quad\quad\quad  \quad\quad\quad\quad \quad\quad\quad\quad   \quad\quad\quad\quad 日期:\quad\quad 年 \quad  月  \quad  日\\
\end{flushleft}
\clearpage
\begin{center}
\heiti{\zihao{2} \textbf{关于使用授权的声明}}
\end{center}

\songti \zihao{-3} \linespread{1.25} \selectfont {本人在指导老师指导下所完成的毕业设计(论文)及相关资料(包括图纸、试验记录、原始数据、实物照片、图片、录音带、设计手稿等),
知识产权归属沈阳建筑大学。本人完全了解沈阳建筑大学有关保存、使用毕业设计(论文)的规定,
本人授权沈阳建筑大学可以将本毕业设计(论文)的全部或部分内容编入有关数据库进行检索,
可以采用任何复制手段保存和汇编本毕业设计(论文)。如果发表相关成果,一定征得指导教师同意,
且第一署名单位为沈阳建筑大学。本人离校后使用毕业毕业设计(论文)或与该论文直接相关的学术论文或成果时,
第一署名单位仍然为沈阳建筑大学。\\}
\begin{flushleft}
    \zihao{-3} 作者签名: \quad\quad\quad\quad  \quad\quad\quad\quad \quad\quad\quad\quad   \quad\quad\quad\quad 日期:\quad\quad 年 \quad  月  \quad  日\\
    \zihao{-3} 导师签名: \quad\quad\quad\quad  \quad\quad\quad\quad \quad\quad\quad\quad   \quad\quad\quad\quad 日期:\quad\quad 年 \quad  月  \quad  日\\ 
\end{flushleft}

\clearpage

%%=============设计(论文)任务书===========
%\begin{center}
%\zihao{-2}\textbf{\songti 本科生毕业设计(论文)任务书} 
%\end{center}
%\smallskip
%\renewcommand{\arraystretch}{1.3}
%\begin{tabular}{lll}
%\zihao{4} \textbf{\songti 学生姓名: 曹宇} & & \zihao{4} \textbf{\songti 专业班级:\quad\quad 船海1006班} \\ 
%\zihao{4} \textbf{\songti 指导教师:徐海祥}&\makebox [3cm] & \zihao{4} \textbf{\songti 工作单位:\quad 武汉理工大学} \\ 
%\end{tabular}\\
%\begin{tabular}{lll}
%\zihao{4} \textbf{\songti 设计(论文)题目:}& \zihao{4} \textbf{\songti  武汉理工本科论文\LaTeX 模板 } &\\ 
%\zihao{4} \textbf{\songti 设计(论文)主要内容:} \\
%\end{tabular} \\ 
%\begin{enumerate}
%\item \LaTeX 环境的配置
%\item 主要字体的控制和数学公式的选用
%\item 图表和代码的粘贴
%\end{enumerate}
%\begin{tabular}{ll}
%\zihao{4} \textbf{\songti 要求完成的主要任务:}
%\end{tabular} \\ 
%\begin{enumerate}
%\item 选择合适的\TeX 编辑系统
%\item 学习如何使用控制代码完成排版
%\item 合理的安排学习和科研的时间来发展自己兴趣爱好
%\end{enumerate}
%\begin{tabular}{ll}
%\zihao{4} \textbf{\songti 必读参考资料:}
%\end{tabular}
%\begin{enumerate}
%\item \LaTeX  \quad User Manual
%\item  字体设计的艺术
%\end{enumerate}
%\begin{tabular}{lll}
%\zihao{4} \textbf{\songti 指导教师签名: }&\makebox [4cm]& \zihao{4} \textbf{\songti 系主任签名:} \\
%& & \zihao{4} \textbf{\songti 院长签名(章)}
%\end{tabular}
%\thispagestyle{empty}
%\clearpage
%==========本科生毕业设计(论文)开题报告  =============
%\begin{center}
%\zihao{-2} \textbf{\songti 武汉理工大学}\\
%\zihao{-2} \textbf{\songti 本科生毕业设计(论文)开题报告} 
%\end{center}
%\begin{tabular}{|l|}
%\hline \rule[-2ex]{0pt}{5.5ex} \makebox[13.5cm][l]{\zihao{4} \heiti 1、目的及意义(含国内外的研究现状分析) } \\ 
%\quad \LaTeX 是国际通行的科技论文排版软件,国际上科研机构和大学都采用它写作\\
%\quad 国内著名高校都有自己的本科生\LaTeX 模板供毕业生使用\\
%\quad 但是武汉理工大学还没有本科生\LaTeX 模板可以参考\\
%\quad 人类的价值在于创造而不是索取 \\
%\hline \rule[-2ex]{0pt}{5.5ex}  \zihao{4} \heiti
%2、基本内容和技术方案\\ 
%\quad 采用GITHUB托管降低代码维护成本\\
%\quad 加入在线\TeX 编辑器的使用简介 \\
%\quad 授人以渔,注重方法和理念的引导\\
%\hline \rule[-2ex]{0pt}{5.5ex}  \zihao{4} \heiti
%3、进度安排 \\ 
%\quad 离 deadline 两个月吃喝玩乐 \\
%\quad 离 deadline 一个月吃喝玩乐 \\
%\quad 离 deadline 半个月吃喝玩乐 \\
%\quad 离 deadline 一个星期狂写论文 \\
%\hline \rule[-2ex]{0pt}{5.5ex} \zihao{4} \heiti
%4、指导教师意见 \\ 
%\quad 曹宇同学是个好同志\\
%\quad 曹宇同志是个好同学\\
%\quad 本表格是支持跨页的长表格,你可以复制上面的内容进行测试\\
%\quad 具体方法是将tabular改为 longtable然后再编译\\
%\makebox[10cm][r]指导教师签名:\\
%\makebox[12cm][r]\quad 年\quad 月\quad 日\\
%\hline 
%\end{tabular} 
%\thispagestyle{empty}

\pagestyle{plain}
\pagenumbering{Roman}
\begin{center}
\section*{\zihao{-2}  \textbf{摘 ~~ 要}}
\end{center}

\vskip0.5cm
本设计名称为“X————————————————”,建筑总高度为 95.25m, 建筑层数为 30  层,主要针对该项目的施工过程进行全面的安全方案设计。
通过制定本工程的施工组织设计,了解各个部分工程的基本施工方案,制定评价单元,从而确定施工过程中人的不安全行为和物的不安全状态,
对施工现场中的危险源进行辨识, 运用事故树、预先危害分析、安全检查表等方法对施工现场中存在的危险源进行评价, 对已经发现的危险危害因素做出预防措施,
并且制定相应的应急预案。

本工程属于框架剪力墙结构,其中脚手架工程采用落地式和悬挑式脚手架两种, 搭设高度均为 18.00m;模板工程采用木模板,支模高度为 8.95m 属于高支模;
脚手架及模板支撑体系均采用 Ф48.3×3.6 钢管;基坑达到 9.40m,采用混凝土灌注桩并配有双层锚杆支护方式。均属于超出一定规模的危险性较大的分部分项工程,
风险性极大, 因此本次设计针对以上三部分分别做出了专项方案。

根据 “安全第一,预防为主,综合治理”的安全方针,建立项目安全生产管理组织机构,健全和完善相关管理制度。根据危险源辨识与评价,制定重大事故的相应应急预案,
形成完整的管理责任流程,为项目部安全管理提供完整、高效的管理依据。\\



{\zihao{-4} \heiti 关键词: 施工组织设计;危险源辨识;安全评价;专项施工方案;应急预案}
\addcontentsline{toc}{section}{摘要}
\pagestyle{fancy}

\clearpage
\begin{center}
    \section*{\zihao{-2}  \textbf{Abstract}}
    \end{center}

   %用了Times New Roman字体来美化观感

   This design is entitled "XX safety construction or ganization and design", building a total height of 95.25 m, building layer number is 30, 
   mainly for the project construction process to conduct a comprehensive safety plan design. Are formulated by the construction organization design, 
   to understand each part project of basic construction plan, make evaluation unit, to determine the construction process of human uns afe behavior 
   and unsafe state of the content of the construction site of the hazards are identif ied, using the fault tree, preliminary hazard analysis, safety 
   check list method to evaluate th e hazards that exist in the construction site, have found that the risk of harm factors to make preventive measures,
    and formulate the corresponding contingency plans.

   This project belongs to the frame shear wall structure, in which the scaffold project adopts fl oor type and overhanging type of scaffolding, 
   and the height of erection is 18.00m. The for mwork adopts wooden template, and the supporting height is 8.95m. Scaffolding and formw ork 
   supports system adopts Ф 48.3 x 3.6 steel tube; The foundation pit reaches 9.40m, with concrete cast-in-place pile and double-deck anchor 
   bolt support. All of them belong to sub-p rojects with greater risks than a certain scale, which have great risks. Therefore, this design 
   makes special plans for the above three parts respectively.

   According to the safety policy of "safety first, prevention first, comprehensive management ", the project safety production management 
   organization is established, and related manage ment system is improved. According to the identification and evaluation of dangerous sources,
    the corresponding emergency plan for major accidents is formulated to form a complete management responsibility flow, providing a complete 
    and efficient management basis for t he safety management of the project department.


\textbf{\zihao{4} Key Words: Construction organization design; Hazard identification; Safety assessment; Special construction plans; The emergency response plan}
\pagestyle{fancy}
\addcontentsline{toc}{section}{Abstract}





\tableofcontents 

%============== 论文正文   =================
\pagestyle{fancy}
\pagenumbering{arabic}
\section{工程概况}
\subsection{施工组织设计编制基本原则}

施工组织设计按照编制对象,可分为施工组织总设计、单位工程施工组织设计。施工组织设计应包括编制依据、工程概况、施工部署、施工准备与资料配置计划、施工进度计划、主要施工方法、施工管理措施、施工现场平面布置等主要内容;施工组织设计的编制必须遵循工程建设程序,并符合下列原则:

\begin{itemize}

    \item [1)] 符合国家有关法律法规、现行规范,符合地方规程、行业标准的要求;

    \item [2)] 满足建筑施工合同或招标文件中关于建筑工程进度、质量、环境保护、职业健康、安全、工程造价等工程管理目标的要求;

    \item [3)] 积极开发、推广运用新技术、新工艺、新材料、新设备;

    \item [4)] 坚持科学的施工程序和合理的施工顺序,做到资源的优化组织和合理配置,采用流水施工和网络计划的方法,实现均衡施工,努力实现科学、合理的经济技术指标;

    \item [5)] 积极响应国家关于低碳、节能、环保方面的方针、政策;采取先进的技术和管理措施,推广建筑节能和绿色施工。

    \item [6)] 与建筑施工单位质量、环境、职业健康安全、项目管理规范四合一标准的有效结合,贯彻质量、环境、职业健康安全管理国家管理规范的要求;

\end{itemize}

\subsection{施工组织设计编制程序}
\subsection{指导方针及编制依据}

\subsubsection{指导方针}
\subsubsection{编制依据}

施工组织设计应以下列内容为主要编制依据:

\begin{itemize}

    \item [1)] 与建筑工程有关的法律、法规和相关文件;

    \item [2)] 国家现行有关标准、规范和技术经济指标;

    \item [3)] 工程所在地的行政主管部门的管理要求;

    \item [4)] 建筑施工行业相关的的质量、环境、职业健康安全管理体系管理规范的要求;

    \item [5)] 工程施工合同及招投标文件;

    \item [6)] 工程设计文件

    \item [7)] 项目周边环境、现场条件、工程地质和水文、气象等自然条件;

    \item [8)] 与工程项目施工有关的资源供应、生产要素配置情况;

    \item [9)] 施工单位的生产能力、机具设备状况、技术水平等等。
    
\end{itemize}

\subsection{工程概况}
\subsection{建筑设计概况}
\subsection{结构设计概况}
\subsection{气象地质特点}


\section{施工组织设计}
\subsection{施工流向、程序及顺序}

项目总体施工顺序按照先地下、后地上;先结构、后围护;先主体、后装修;先土建、后专业的总施工顺序原则进行部署。
主体工程自下而上施工,室内装修采用自上而下的流向,水、电、电梯和设备等各专业分项工程在结构阶段配合结构施工做好预埋及预留的同步作业,
其施工阶段随结构与装修工程穿插进行,专业分项工程与土建工程必须相互密切配合,由项目部统一协调与指挥,确保工程顺利进行。

对于与场内外有联系的一些工程,如道路工程、管线工程等,其施工应从场外开始,然后再逐步向场内延伸。

先完成全场性的工程,然后再完成各独立的建筑物和构筑物。

在施工时应先完成零点标高以下的工程,然后再完成零点标高以上的部分。施工时应贯彻先深后浅的原则,即先做深层的,再做浅层的。一层一层的做上来,
只有在完成零点标高以下的工程之后,再进行地面以上工程的施工。

主要的施工流向和顺序详见 \ref{fig:c2f3}

\begin{figure}[thbp!]
    \centering
    \includegraphics[width=1.0\linewidth]{figure/c2f3.png}
    \caption{主要施工顺序图}
    \label{fig:c2f3}
\end{figure}

\subsection{施工组织机构及主要管理人员职能}
\subsubsection{施工组织机构}

本工程拟实行项目法施工管理,委派实践经验丰富和管理水平高的同志担任项目部主要负责人,选聘技术、管理水平高的技术人员、管理人员、专业工长组建项目部。

项目管理层由项目经理、项目副经理、技术负责人、安全主管、质量主管、材料主管、保卫主管、机械主管和后勤主管等成员组成,
在建设单位、监理单位和公司的指导下,负责对本工程的工期、质量、安全、成本等实施计划。组织、协调、控制和决策,对各生产施工要素实施全过程的动态管理。

项目经理部对工程项目进行计划管理。计划管理主要体现在工程项目综合进度计划和经济计划。

作业层人员的配备:施工人员均挑选有丰富施工经验和劳动技能的正式工和合同工,分工种组成作业班组,挑选技术过硬、思想素质好的正式职工带班。

为保证项目部管理层指令畅通有效,工作安排采用“施工任务书”的形式。要求签发人和执行人签字,项目经理层作为执行的监督者。
施工任务书的工作内容完成后由签发人封闭并签字,如未能封闭必须找出原因并对执行人进行处罚。

项目经理部组织机构图见 \ref{fig:c2f1}

\begin{figure}[thbp!]
    \centering
    \includegraphics[width=0.7\linewidth]{figure/c2f1.png}
    \caption{施工组织机构图}
    \label{fig:c2f1}
\end{figure}

\subsubsection{主要管理人员职责}

(1) 项目经理岗位职责\\

\quan{1} 全面负责本工程的一切事务,认真贯彻执行《建筑法》、《合同法》和国家有关劳动保护法令和制度以及公司的各项管理制度,
贯彻安全第一、预防为主的方针,按规定搞好安全防范措施,把安全工作落到实处,在各种经济承包中必须包括安全生产、做到讲效益必须讲安全,抓生产首先必须抓安全;

\quan{2} 认真熟悉施工图纸、组织编制施工组织设计方案和施工安全技术措施,组织编制工程总进度计划表和月进度计划表及各施工班
组的月进度计划表。会同项目部相关人员精选强有力的施工队伍,编制工程进度计划及人力、物力计划和机具、用具、设备计划,做到合理组织施工,如发现计划工期无法保证应及时进行调整;

\quan{3} 制定适合本工程项目的管理细则、方案及措施,组织项目部例会,合理安排、科学引导、顺利完成本工程的各项施工任务。
对项目质量全面负责,负责制定质量考核目标,并监督检查;

\quan{4} 对项目成本控制负责,安排、搞好项目的成本核算(按单项和分部分项)单独及时核算,并将核算结果及时通知公司各部门部的管理人员,
以便及时改进施工计划及方案,争创更高效益。及时向各班组下达施工任务书及材料限额领料单;

\quan{5} 对现场文明施工管理负责,组织制定现场文明各项施工管理规定。根据本工程施工现场情况合理规划布局现场平面图,安排、实施、创建文明工地。要求布局合理、经济;

\quan{6} 深入实际了解员工的生活工作学习情况,采纳员工中的合理化建议,妥善解决好施工中出现的各类问题,保质保量顺利完成本工程施工任务。\\

(2) 施工员岗位职责\\

\quan{1} 在项目经理的直接领导下开展工作,贯彻安全第一、预防为主的方针,按规定搞好安全防范措施,把安全工作落到实处,做到讲效益必须讲安全,抓生产首先必须抓安全;

\quan{2} 认真熟悉施工图纸、编制各项施工组织设计方案和施工安全、质量、技术方案,编制各单项工程进度计划及人力、物力计划和机具、用具、设备计划。并监督计划执行情况;

\quan{3} 组织新进场员工技术培训,并做好每道工序的技术交底工作;

\quan{4} 编制文明工地实施方案,根据本工程施工现场合理规划布局现场平面图,安排、实施、创建文明工地;

\quan{5} 编制工程总进度计划表和月进度计划表及各施工班组的月进度计划表,并跟踪实施情况;

\quan{6} 向各班组下达施工任务书及材料限额领料单。配合项目经理工作。\\

(3) 安全员岗位职责\\

\quan{1} 在项目经理领导下,全面负责监督实施施工组织设计中的安全措施、并负责向作业班组进行安全技术交底;

\quan{2} 检查施工现场安全防护、用电安全、机械设施、等是否符合安全规定和标准。如发现施工现场有不安全隐患,应及时提出改进措施,
督促实施并对改进后的设施进行检查验收。对不改进的,提出处置意见报项目负责人处理;

\quan{3} 正确填报施工现场安全措施检查情况的安全生产报告,定期提出安全生产的情况分析报告的意见;

\quan{4} 处理一般性的安全事故;

\quan{5} 按照规定进行工伤事故的登记,统计和分析工作;

\quan{6} 同各施工班组及个人签订安全生产协议书;

\quan{7} 随时对施工现场进行安全监督、检查、指导,并做好安全检查记录。对不符合安全规范施工的班组及个人进行安全教育、处罚,并及时责令整改;

\quan{8} 在安全检查工作中不深入、不细致及存在问题不提出意见又不向上级汇报,所造成的责任事故,应承担全部责任及后果。\\

(4) 质检员岗位职责\\

\quan{1} 在项目经理领导下,负责检查监督施工组织设计的质量保证措施的实施,组织建立各级质量监督保证体系;

\quan{2} 严格监督进场材料的质量、型号、规格、监督各项施工班组操作是否符合规程;

\quan{3} 按照规范规定的分部分项检验方法和验收评定标准,正确进行自检和实测实量,填报各项检查表格,
对不符合工程质量标准、质量要求返工的分部分项工程,写出返工意见并出具罚款单。保证按规定对每一层次、工序不漏检,
并配合施工进度不影响施工。凡经检验不合格的工程,应及时报告项目经理以便采取措施;

\quan{4} 提出工程质量通病的防治措施,提出制订新工艺、新技术的质量保证措施建议;

\quan{5} 对工程的质量事故进行分析,提出处理意见;

\quan{6} 向每个施工班组做(质量验收评定标准)交底;

\quan{7} 每个房间施工都应在施工段(墙、柱、梁、板)贴上质量检查验收表。只有经检验合格才能进入下一工序施工。 \\

\subsubsection{主要管理人员安全生产岗位责任制}

(1) 项目经理安全生产岗位责任制\\

\quan{1} 工程项目经理对工程项目的安全生产负有全面责任;

\quan{2} 建立和健全项目体安全管理网络,确定安全管理目标,组织编制安全保证计划;

\quan{3} 根据工程特点,加强对分包单位的控制和管理;

\quan{4} 认真执行各项安全生产规章制度,明确项目体各管理岗位的安全生产职责,负责检查项目体的安全生产责任制落实情况;

\quan{5} 适时组织对工程项目部的安全体系评审和协调;

\quan{6} 落实安全保证计划的资源配置;

\quan{7} 依据施工组织设计,落实各项安全技术措施,对分部分项工程必须派人进行安全技术交底;

\quan{8} 落实专人负责检查特种作业人员持证情况,对新的分包单位进入工地,要有针对地进行安全教育,制止违章操作;

\quan{9} 发生工伤事故成立即时组织抢救,保护现场,迅速如实上报,并参加事故调查处理,按“四不放过”的原则,落实各项整改工作;

\quan{10} 自觉接受上级安全生产部门的监督和管理。\\

(2) 安全员安全生产岗位责任制\\

\quan{1} 贯彻执行劳动保护法规、制度,做好安全生产的宣传,教育和管理工作;

\quan{2} 贯彻安全保证计划中的各项安全技术措施,组织参与安全设施、施工用电、施工机械的验收;

\quan{3} 对进入现场使用的各种安全用品及机械设备进行各项必要材料的存档及抽查工作;

\quan{4} 对分包单位进行安全、消防检查,指导或协助生产负责人解决生产中不安全问题;

\quan{5} 参加组织安全、消防活动和安全、消防检查,制止违章作业。对事故隐患开具整改单限期整改并复验,
对有严重险情的有权决定轶作业并及时上报。对违反劳动保护、安全生产法规的行为,经说服劝阻无效时,有权越级上报或举报;

\quan{6} 督促有关分包单位做好对新工人、特殊工种人员以及换岗人员的安全技术教育,做好个人安全档案工作;

\quan{7} 对施工组织设计中的安全技术措施在实施过程中进行督促检查;

\quan{8} 负责对分包单位安全生产的责任考核工作;

\quan{9} 参加伤亡事故调查、分析、处理工作。按照“四不放过”的原则,做好安全事故的统计、分析和报告以及归档工作。\\

(7) 项目部生产班组长岗位责任制\\

\quan{1} 按照施工方案,组织劳动力进场,彻底做好班组的施工工艺和安全技术措施交底工作;

\quan{2} 监督、检查本班组操作工人按图纸、规范、施工方案施工;

\quan{3} 组织班组进行自检、互检和交接检工作,发现不合格项目及时组织工人进行整改,确保本班组工作面的质量符合标准;

\quan{4} 负责传达项目部的各项管理内容和上报班组各项情况,及时进行调解;

\quan{5} 认真遵守安全规章和有关安全生产制度,对本组人员在生产中的安全健康负责;

\quan{6} 搞好安全活动日,开好班前、班后安全会,对新调入的工人进行现场班组级安全教育;

\quan{7} 组织本班级职工学习施工技术和安全规程及制度,检查执行情况,在任何情况下,均不得违章,不得擅自动用机械、电气、架子等设备;

\quan{8} 经常检查施工现场的安全生产情况,加强安全自检,发现问题及时解决,不能解决的采取措施并及时上报;

\quan{9} 发生工伤事故要详细记录并及时上报,组织全组人员认真分析,提出防范措施。发生重大伤亡事故要保护现场并立即上报项目部主管。\\

\subsection{施工总平面布置说明}

施工总平面图布置图的设计和布置应该符合以下原则:

(1) 施工平面布置应严格控制在建筑红线之内,平面布置要紧凑合理,尽量减少施工用地且应尽量利用原有建筑物或构筑物。

(2) 合理组织运输,保证现场运输道路畅通,尽量减少二次搬运。

(3) 各项施工设施布置都要满足方便施工、安全防火、环境保护和劳动保护的要求。

(4) 在平面交通上,要尽量避免土建、安装以及其他各专业施工相互干扰。

(5) 平面布置应符合施工现场卫生及安全技术要求和防火规范。

(6) 结合拟采用的施工方案及施工顺序。

(7) 满足半成品、原材料、周转材料堆放及钢筋加工需要。

(8) 满足不同阶段、各种专业作业队伍对宿舍、办公场所及材料储存、加工场地的需要。

(9) 各种施工机械既满足各工作面作业需要又便于安装、拆卸。

\subsubsection{现场道路}

根据各加工厂、仓库及各施工对象的相对位置,研究货物转运图,区分主要道路和次要道路,进行道路的规划。规划厂区内道路时,应考虑以下几点,

\quan{1} 合理规划临时道路与地下管网的施工程序。在规划临时道路时,应充分利用拟建的永久性道路,提前修建永久性道路或者先修路基和简易路面,
作为施工所需的道路,以达到节约投资的目的。若地下管网的图纸尚未出全,必须采取先施工道路,后施工管网的顺序时,
临时道路就不能完全建造在永久性道路的位置,而应尽量布置在无管网地区或扩建工程范围地段上,以免开挖管道沟时破坏路面;

\quan{2} 保证运输通畅。道路应有两个以上进出口,道路末端应设置回车场地,且尽量避免临时道路与铁路交叉。
厂内道路干线应采用环形布置,主要道路宜采用双车道,宽度不小于 6m,次要道路宜采用单车道,宽度不小于 3.5m;

\quan{3} 选择合理的路面结构。临时道路的路面结构,应当根据运输情况和运输工具的不同类型而定。一般场外与省、市公路相连的干线、因其以后会成为永久性道路,
因此,一开始就建成混凝土路面;场区内的干线和施工机械行驶路线,最好采用碎石级配路面,以利修补。场内支线一般为土路或砂石路。

\subsubsection{现场材料堆放}

仓库的布置较灵活。一般中心仓库布置在工地中央或靠近使用的地方,也可以布置在靠近于外部交通连接处。砂石,水泥、石灰木材等仓库或堆场宜布置在搅拌站、
预制场和木材加工厂附近;砖、瓦和预制构件等直接使用的材料应该直接布置在施工对象附近,以免二次搬运。工业项目建筑工地还应考虑主要设备的仓库(或堆场),
一般笨重设备应尽量放在车间附近,其他设备仓库可布置在外围或其他空地上。

\subsubsection{现场临时水电布置}

当有可以利用的水源、电源时,可以将水电从外面接入工地,沿主要干道布置干管、主线,然后与各用户接通。临时总变电站应设置在高压电引入处,不应放在工地中心;
临时水池应放在地势较高处。当无法利用现有水电时,为了获得电源,可在工地中心或工地中心附近设置临时发电设备,沿干道布置主线;为了获得水源可以利用地上水或地下水,
并设置抽水设备和加压设备(简易水塔或加压泵,,以便储水和提高水压。然后把水管接出,布置管网。施工现场供水管网有环状、枝状和混合式三种形式,根据工程防火要求,
应设立消防站,一般设置在易燃建筑物(木材、仓库等)附近,并须有通畅的出口和消防车道,其宽度不宜小于 6m,与拟建房屋的距离不得大于 25m,也不得小于 5m,沿道路布置消防栓时,
其间距不得大于 100 ,消防栓到路边的距离不得大于 2m。

临时配电线路布置与水管网相似。工地电力网 3-1OkV 的高压线采用环状,沿主干道布置;380/220V 低压线采用枝状布置。工地上通常采用架空布置,距路面或建筑物不小于 6m。

\subsubsection{现场生活区与行政区布置}

行政与生活临时设施包括:办公室、汽车库、职工休息室、开水房、小卖部、食堂、俱乐部和浴室等。
根据工地施工人数,可计算这些临时设施的建筑面积。应尽量利用建设单位的生活基地或其他永久建筑,不足部分另行建造。

一般全工地性行政管理用房宜设在全工地入口处,以便对外联系;也可设在工地中间,便于全工地管理。
工人用的福利设施应设置在工人较集中的地方,或工人必经之处。生活基地应设在场外,距工地 500-1000m 为宜。食堂可布置在工地内部或工地与生活区之间。

\subsection{施工总进度计划及工期保证措施}
\subsubsection{整体工期控制目标}

针对关键路径上的各分项工程制定进度控制计划,各流水段制定流水计划,每月、每周均制定进度计划加以控制,每天通过工程例会监督进度计划落实情况,加以动态管理。 

根据施工进度计划制定具体的材料采购及进场计划,详细了解生产周期、运输周期、可能的影响因素,确保材料的生产、运输、进场、检验各环节均处于受控状态,
在合同条款中对材料拖期到货明确相应罚则。

每日落实进度计划,对工期影响因素做好提前的沟通解决,当出现严重影响工程总体进度的情况,及时向公司汇报,并协调以全力解决,
并适当调整进度计划以确保项目总进度按期完成。\cite{zhang_bim-based_2015}

项目施工的管理流程见图 \ref{fig:c2f2}

\begin{figure}[thbp!]
    \centering
    \includegraphics[width=0.8\linewidth]{figure/c2f2.png}
    \caption{项目施工管理流程}
    \label{fig:c2f2}
\end{figure}

\clearpage
\subsubsection{工期保证措施}

为确保本工程如期、保质保量地完成,项目部将在以下几个方面采取相应的措施:\\

(1) 施工组织保障措施\\

为保证施工计划完成,我们将选派有丰富的现场施工组织管理经验的、并曾担任过类似工程的项目经理担任该工程项目经理。
以加强项目部管理能力和组织协调能力,及时解决施工中遇到的问题,保证施工中各环节、各专业、各工种之间的协调与平衡。
对工程质量、安全进行监督及人力、财力、物力的统一调度,确保施工顺利进行,杜绝因管理不善造成施工脱节、资源浪费、工期延误现象的发生。\\

(2) 合理的施工方案\\

\quan{1} 充分熟悉本工程的设计图纸,对拟定的施工组织设计、施工方案及方法进行认真的分析比较,作到统筹组织、全面安排,确保总体目标计划。在施工过程中制定阶段性工期控制点,确保按期完工。针对工程特点,采用分段流水施工方法,减少技术间歇,突出重点,制定严密的、紧凑的、合理的施工穿插,尽可能压缩工期,加快施工进度;

\quan{2} 合理地加入投入机械,提高机械化作业程度,充分满足工程所需的人、财、物要求;

\quan{3} 对各班组进行教育,责任到人。各区分段划分负责人,举行施工质量、进度、安全等评比,奖优罚劣。\\

(3) 做好各种资源的供应\\

\quan{4} 根据施工组织设计的要求和施工进度计划中各个阶段控制点的要求,编制劳动力进场计划、材料进场计划、机械设备进场计划、资金使用计划,以保证各种资源能满足施工需要;

\quan{5} 物资材料计划有明确的材料数量、规格和进场时间,现场材料储备应有一定的库存量,以保证工程进度提前或节假日运输困难时工程对物资的需要,确保现场施工正常进行;

\quan{6} 劳动力进场要保证质量,工人进场前必须进行严格的培训和考核。保证足够数量的劳动力;

\quan{7} 按照计划施工机械进场前对机械进行必要的维护、保养和试运转工作,保证所有机械进场后能够投入正常使用。使用期间加强维护与保养,确保机械能正常运行;

\quan{8} 保证足够的资金用于施工,专款专用,确保本工程正常运转。\\

(4) 严格的管理与控制\\

\quan{1} 强化项目方法管理,推行项目方法施工,实行项目经理负责制,设立能协调各方面关系的调度指挥机构,配备素质高、能力强,有开拓精神的管理班子,确保施工进度;

\quan{2} 全面推行计划动态管理,控制工程进度,建立主要形象进度控制点,运用网络计划跟踪技术和动态管理方法,做到周保旬,旬保
月,坚持月平横,周调度、工期倒排,确保总进度的计划实施;

\quan{3} 认真做好施工中的计划统筹、协助与控制。严格坚持落实每周工地施工例会制度,做好每日工程进度安排,确保各项计划落实。
编排详细的工程施工总进度计划,并采用微机管理技术,对施工计划实行动态管理;建立主要工程进度控制点,
围绕总进度计划,编制月、周施工进度计划,作到各分部分项工程的实际进度按计划要求进行;每期根据前期完成情况和其他预测情况变化,
对当期计划和后期计划、总计划进行重新调整和部署,确保按原定或因非施工原因调整了的期限交工;

\quan{4} 实行奖励机制,拟定拿出一定的资金作为目标管理和科技进步奖励基金,充分调动全体施工人员的积极性和创造性,力保各项目标按期实现;

\quan{5} 制定各工序的操作规程和质量标准,强化施工现场管理,做到安全文明施工,努力实现施工管理的标准化、科学化、合理化,使施工有条不紊;

\quan{6} 强化项目部内部管理人员效率与协调,增强与业主的联系,加强对劳务人员的控制和与各供货商的协作,并确保各方及个人的职责分工,
减少扯皮现象,争取将围绕本工程建设的各方面人员充分调动起来,共同完成工期总目标;

\quan{7} 创造和保持施工现场各方面各专业之间的良好的人际关系,使现场各方认清其间的相互依赖和相互制约的关系。特别是加强同交通疏导、
材料运输、周围居民的协调,增进与业主、监理、设计单位的联系和配合,及时解决问题。

\subsection{主要项目施工方法和技术措施}
\subsubsection{土方工程}

\begin{itemize}
    \item [1)]场地清理后,即可进行土方开挖,开挖顺序为:按已划分的二个施工流水段分别同时开挖,就地堆积余土。
    \item [2)]基坑开挖按确定的施工顺序逐个采用人工开挖,人工装土,自卸车运土的方式施工。
    \item [3)]根据工程标底的要求:土方开挖为三类土,为保证土壁稳定,开挖时放坡系数暂为 1:0.33 ,由于无现成地质资料参考暂不考虑基坑护壁的支撑。
    \item [4)]土方开挖先测放基坑开挖白线,人工开挖至基底标高后,进行修正清理,测量放线人员准确测放基底标高、轴线、基础的外形尺寸,经自验无误后,做好基坑隐蔽记录,并请监理工程师复核。
    \item [5)]基坑回填采用人工夯实。回填注意控制土含水率,采用同类土填筑,分层回填虚铺厚度每层控制在 30cm 以内,夯实后,干容重不得小于 1.65$\mathsf{g/cm^3}$。
    \item [6)]基坑开挖程序一般是:测量放线 → 切线分层开挖 → 排降水 → 修坡 → 整平 → 留足预留土层等。相邻基坑开挖时,应遵循先深后浅或同时进行的施工程序。挖土应自上而下水平分段分层进行,每层 0.3m 左右,边挖边检查坑底宽及坡度,不够时及时修整,每3m左右修一次坡,至设计标高,再统一进行一次修坡清底,检查坑底宽和标高,要求坑底凹凸不超过1.5m。
    \item [7)]基坑开挖应防止对地基土的扰动。采用机械开挖,为避免破坏基底土,应在基底标高以上预留一层人工清理。
    \item [8)]基坑开挖完后应进行验槽,作好记录,如发现地基土质与地质勘探报告、设计要求不符时,应与有关人员研究及时处理。
\end{itemize}

\subsubsection{钢筋工程}

(1) 钢筋制作\\

钢筋加工制作时,要将钢筋加工下料表与设计图复核,检查下料表是否有错误和遗漏,对每种钢筋要按下料表检查是否达到要求,经过这两道检查后,
再按下料表放出实样,试制合格后方可成批制作,加工好的钢筋要挂牌堆放整齐有序。

施工中如需要钢筋代换时,必须先充分了解设计意图和代换材料性能,严格遵守现行钢筋混凝土设计规范的各种规定,并不得以等面积的高强度钢筋代换低强度的钢筋。
凡重要部位的钢筋代换,须征得设计单位同意,并有书面通知时方可代换。

\quan{1} 钢筋表面应洁净,粘着的油污、泥土、浮锈使用前必须清理干净,可结合冷拉工艺除锈;

\quan{2} 钢筋调直,可用机械或人工调直。经调直后的钢筋不得有局部弯曲、死弯、小波浪形,其表面伤痕不应使钢筋截面减小 5\%;

\quan{3} 钢筋切断应根据钢筋号、直径、长度和数量,长短搭配,先断长料后断短料,尽量减少和缩短钢筋短头,以节约钢材。\\


(2)钢筋绑扎与安装\\

钢筋绑扎前先认真熟悉图纸,检查配料表与图纸,设计是否有出入,仔细检查成品尺寸、形状是否与下料表相符。核对无误后方可进行绑扎。
采用 20# 铁丝绑扎直径 12 以上钢筋,22# 铁丝绑扎直径 10 以下钢筋;

\quan{1} 竖向钢筋的弯钩应朝向柱心,角部钢筋的弯钩平面与模板面夹角,对矩形柱应为 45° 角,截面小的柱,用插入振动器时,弯钩和模板所成的角度不小于 15°;

\quan{2} 箍筋的接头应交错排列垂直放置;箍筋转角与竖向钢筋交叉点均应扎牢(箍筋平直部分与竖向钢筋交叉点可每隔一根互成梅花式扎牢)。绑扎箍筋时,铁线扣要相互成八字形绑扎;

\quan{3} 柱筋绑扎时应吊线控制垂直度,并严格控制主筋间距。柱筋搭接处的箍筋及柱立筋应满扎,其余可梅花点绑扎;

\quan{4} 当梁高范围内柱(墙)纵筋斜度 b/a≤1/6 时,可不设接头插筋;当 b/a>1/6 时,应增设上下柱(墙)纵筋的连接插筋,锚入柱(墙)内。 \\

(3)质量标准\\

\quan{1} 钢筋的材质。规格及焊条类型应符合钢筋工程的设计和施工规范,有材质及产品合格证书和物理性能检验,对于进口钢材需增加化学性能检定,检验合格后方能使用;

\quan{2} 钢筋的规格、形状、尺寸、数量、间距、锚固长度、接头位置、保护层厚度必须符合设计要求和施工规范的规定;

\quan{3} 焊工必须持相应等级焊工证才允许上岗操作;

\quan{4} 在焊接前应预先用相同的材料、焊接条件及参数,制作二个抗拉试件,其试验结果大于该类别钢筋的抗拉强度时,才允许正式施焊,此时可不再从成品抽样取试件。


\subsubsection{模板及支撑工程}

柱模板安装前,应按设计要求绑扎好柱筋,做好隐蔽记录,按照柱脚轴线及断面尺寸装好柱脚外围拦板,再扶柱板,扶直后先用临时支撑固定,待一列(纵横方向)柱模板装好,
进行中轴线和垂直度校正,校正无误后四周用斜撑钉牢固。

墙自身固定均采用竖向木枋(@≤300)和水平Φ48钢管(间距同螺杆)组成,用 Φ16 螺栓水平钢管对拉以控制截面,螺杆起步间距 ≤250,横向间距为 ≤500mm,竖向间距 ≤500,
楼层上预埋 Φ20@500 的钢筋定位桩以定位并防止墙根部浇混凝土时移位。

外墙模板的空间固定采用顶撑相结合的方法固定,即钢支撑作为压杆,钢丝绳花蓝螺杆作拉杆,压杆与拉杆的间距为 1.8m,内墙模板的空间固定采用钢支撑在墙两侧,
斜向对顶的方法固定。为了防止外墙根部上下层接头位置胀模,上层模板应落下并低于下层楼面不小于 300,并支撑于架设在下层柱顶预埋螺栓上的枋木面上。

附墙柱用胶合板,自身固定用坚向木枋和水平钢管,Ф16 对拉螺杆控制截面(竖向 @≤500)。墙柱支模前根据楼层放线先用 30 宽 18 厚胶全板条在砼楼面上钉出墙模板位置,
这样既便于柱模板定位准确,又便于加强柱模板根部固定,防止柱根部混凝土漏浆。

对通排柱模板,应先装两端柱模板,校正固定,拉通长线校正中间各柱模板。

对截面大于	400mm 的柱,用螺栓或钢筋箍收紧柱模木枋,以不鼓突、不漏浆为准。安装柱模时应在下脚留清扫口。


\subsubsection{混凝土工程}

(1) 浇筑的一般要求\\

\quan{1} 浇筑前应对模板浇水湿润,柱模板的清扫口应在清除杂物及积水后再封闭;

\quan{2} 混凝土自吊斗口下落的自由倾落高度不得超过 2 米,如超过 2m 时必须采取加串筒措施;

\quan{3} 浇筑竖向结构混凝土时,如浇筑高度超过3m时,应采用串筒、导管、溜槽或在模板侧面开门子洞;

\quan{4} 浇筑混凝土时应分段分层进行,每层浇筑高度应根据结构特点、钢筋疏密决定。一般分层高度为插入式振动器作用部分长度的 1.25 倍,大不超过 500mm,平板振动器的分层厚度为 200mm;

\quan{5} 使用插入式振动器应快插慢拔,插点要均匀排列,逐点移动,按顺序进行,不得遗漏,做到均匀振实。移动问距不大于振动棒作用半径的 1.5 倍(一般为 300\textasciitilde400mm);
振捣上一层时应插入下层混凝土面 50mm,以消除两层间的接缝。平板振动器的移动间距应能保证振动器的平板覆盖已振实部分边缘;

\quan{6} 浇筑混凝土应连续进行。如必须间歇,其间歇时问应尽量缩短。并应在前层混凝土初凝之前,将次层混凝土浇筑完毕。问歇的最长时间应按所有水泥品种及混凝土初凝条件确定
一般超过 2 小时应按施工缝处理;

\quan{7} 浇筑混凝土时应派专人经常观察模板钢筋、预留孔洞、预埋件、插筋等有无位移变形或堵塞情况,发现问题应立即停止浇灌,并应在已浇筑的混凝土上初凝前修整完毕;

\quan{8} 混凝土浇筑前准确掌握天气情况,避开雨天,尽量安排混凝土在夜间浇筑,以降低较厚处混凝土内部水化热。浇注混凝土时,模板、钢筋、应设专人值班,如有位移、变形应及时处理,确保混凝土质量。\\

(2) 梁、板、柱混凝土浇筑方法\\

\quan{1} 柱、墙浇筑前,或新浇混凝土与下层混凝土结合处,应在底面上均匀浇筑 50mm 厚与混凝土配比相同的水泥砂浆。砂浆应用铁铲入模,不应用料斗直接倒入模内;

\quan{2} 柱、墙混凝土应分层浇筑振捣,每层浇筑厚度控制在 500mm 左右。混凝土下料点应分散布置循环推进,连续进行;构造柱混凝土应分层浇筑,每层厚度不得超过 300mm;

\quan{3} 浇筑墙体洞口时,要使洞口两侧混凝土高大体一致。混凝土振捣要均匀密实,特别是墙厚较小,门窗洞口结构加筋与连接交错钢筋较密的部位,应采用 Φ25 振动棒,
其它墙梁部位采用 Φ50 振动棒,考虑到墙窗洞下墙体位混凝土封模后无法直接振捣,可事先将窗洞下口留成活口,待混凝土浇至该位置并振捣密实后再行封模和加固。
振捣时,振动棒应距洞边 300mm 以上,并从两侧同时振捣,以防止洞口变形。大洞口下部模板应开口并补充振捣;

\quan{4} 肋形楼板的梁板应同时浇筑,浇筑方法应由一端开始用" 赶浆法"推进,先将梁分层浇筑成阶梯 ,当达到楼板位置时再与板的混凝土一起浇筑;

\quan{5} 楼板浇筑的虚铺厚度应略大于板厚,用平板振动器垂直浇筑方向来回振捣。注意不断用移动标志或插杆检查以控制混凝土板厚度。振捣完毕,用刮尺或拖板抹平表面;

\quan{6} 在浇筑与柱、墙连成整体的梁和板时,应在柱和墙浇筑完毕后停歇 1\textasciitilde1.5 小时,使其获得初步沉实,再继续浇筑;

\quan{7} 施工缝设置:宜沿着次梁方向浇筑楼板,施工缝应留置在次梁跨度 1/3 范围内,施工缝表面应与次梁轴线或板面垂直。单向板的施工缝留置在平行于板的短边的任何位置;

\quan{8} 施工缝应用木板、钢丝网挡牢。施工缝处须待已浇混凝土的抗压强度不少于 1.2MPa 时,才允许继续浇筑;

\quan{9} 在施工缝处继续浇筑混凝土前,混凝土施工缝表面应凿毛,清除水泥薄膜和松动石子,并用水冲洗干净。排除积水后,先浇一层水泥浆或与混凝土成分相同的水泥砂浆然后继续浇筑混凝土。\\

(3) 楼梯混凝土浇筑\\

\quan{1} 楼梯段混凝土自下而上浇筑。底板混凝土与踏步混凝土一起浇筑,不断连续向上推进;

\quan{2} 楼梯混凝土宜连续浇筑完成;

\quan{3} 施工缝位置:根据结构情况可留设于楼梯平台板跨中或楼梯段 1/3 范围内; \\

(4) 混凝土的养护\\

\quan{1} 混凝土浇筑完毕后,应在 12 小时以内加以覆盖,并浇水养护;

\quan{2} 混凝土浇水养护日期,掺用缓凝型外加剂的混凝土不得小于 14 天。在混凝土强度达到 1.2MPa 之前,
不得在其上踩踏或施工振动。柱拆模后,用棉布包住,浇水在棉花布上养护,以确保立面结构表面保持湿润状态;

\quan{3} 每日浇水次数应能保持混凝土处于足够的润湿状态。


\subsubsection{脚手架工程}

(1)  钢管采用外径 48mm、壁厚 3.5mm 的焊接钢管,也可采用同样规格的无缝钢管或外径 51mm 、壁厚 3mm 的焊接钢管,钢管材质宜使用力学性能适中的 Q235 钢,
其材性应符合《碳素结构钢》(CB700—88)的相应规定。用于立杆、大横杆、剪力撑和斜杆的钢管长度为 4~6.5m,用于小横杆的钢管长度为 1.8~2.2m,以适应脚手架宽的需要。

(2)  作为脚手架杆件使用的钢管必须进行防锈处理:即对购进的钢管先行除锈,然后内壁擦涂两道防锈漆,处壁涂防锈漆一道和面漆两道。
在脚手架使用一段时间以后,由于防锈层会受到一定的损伤,因此需重新进行防锈处理。

(3)  扣件式脚手架的作业层面可根据所用脚手板的支承要求设置横向平杆,因而可使用各种形式的脚手板。

(4) 对脚手板的技术要求为:

\quan{1} 脚手板的厚度不宜小于 50mm,宽度不宜小于 200mm,重量不宜大于 30kg;

\quan{2} 确保材质符合规定;

\quan{3} 不得有超过允许的变形和缺陷。


\subsubsection{砌体工程}

(1) 砌筑砖砌体时,砖应提前1—2d浇水湿润,含水率宜为10—15\%。

(2) 砖墙砌筑应上下错缝,内外搭砌,灰缝平直,砂浆饱满,水平灰缝厚度和竖向灰缝宽度一般为 10mm,但不应小于 8mm,也不应大于 12mm。

(3) 砖墙的转角处和交接处应同时砌筑,均应错缝搭接,所有填充墙在互相连接、转角处及与混凝土墙连接处均应沿墙高设置 2Φ6@500 通长拉结筋。
对不能同时砌筑而又必须留置的临地问断处应砌成斜槎。如临时间断处留斜槎确有困难时,除转角处外,也可留直槎,但必须做成阳槎,
并加设拉结筋,拉结筋的数量按每 12cm 墙厚原放置一根直径 6mm 的钢筋,间距沿墙高不得超过 50cm,埋入长度从墙的留槎处算起,每边均不应小于50cm,未端应有90° 弯钩。

(4) 隔墙和填充墙的顶面与上部结构接触处用侧砖或立砖斜砌挤紧。

\section{危险因素辨识与评价}

\subsection{危险因素辨识目的和范围}

根据规范《重大危险源辨识》(GB 18218-2018),所谓重大危险源就是指长期或临时地生产、加工、搬运或贮存危险物质,
且危险物质的数量等于或大于临界量的单元对于路桥工程施工过程中,对于重大危险源的界定主要包括:
人的危险行为及管理的漏洞、物的不安全的状态、恶劣的环境影响等,由于建筑施工现场的复杂性,工程施工事故可能随时发生,
并可导致人员死亡及伤害、破坏、财产损失,这对于建筑施工工程的整体施工进度已经企业的经济效益都会造成恶劣的影响,
甚至危及企业的发展,因此建筑施工过程中对于重大危险源的辨识、评价和控制,就显得格外重要。

对于重大危险源的辨识,可根据对危险源危险等级的评定方法进行,一般是对施工过程中危险源带来的风险进行评价分析,
根据评价结果又针对性地进行风险控制,从而达到持续改进的目的。常用的风险评价方法有:
作业条件危险性评价法(LEC)、矩阵法、预先危害分析 (PHA)、故障类型及影响分析 (FMEA)、风险概率评价法 (PRA)、危险可操作性研究 (HAZOP)、事故树分析 (ETA) 等等。

\subsection{危险源辨识}
\subsubsection{基坑工程危险源辨识}
\subsubsection{钢筋工程危险源辨识}

\quan{1} 钢筋加工操作人员未经过相关技术交底和培训,不了解钢筋加工机械的操作方法,导致钢筋加工过程中造成机具伤害;

\quan{2} 钢筋作业人员未取得从业资格证件,无特种作业操作证,在焊接过程中误操作导致触电或者火灾;

\quan{3} 预应力钢筋材料进场未经过检验,导致材料不合格;

\quan{4} 钢筋加工机械未经检验或未按时保养,造成在使用过程中造成触电;

\quan{5} 当使用卷扬机进行冷拉操作时,由于卷扬机固定未有妥善导致被拔出而造成的物体打击;或是卷扬机钢丝绳破损而造成的伤害;

\quan{6} 钢筋切断机锯盘无安全防护导致的机械伤害;

\quan{7} 由于未控制钢筋的冷拉率而造成的的超拉现象,导致机械伤害;

\quan{8} 采用切割机进行切割作业时由于无挡板造成火星飞溅而引起火灾;

\quan{9} 带电操作时未执行“一机一闸一箱一漏”的管理规定而导致储电。

\subsubsection{模板工程危险源辨识}

\quan{1} 由于回填土没压实或压实不到位而造成地基不均匀沉降而导致的模板坍塌;

\quan{2} 由于回填土地基未浇筑硂层或未设置垫板而导致的结构问题或模板坍塌;

\quan{3} 由于立杆、水平杆等支架未按规范搭设而导致的整体支架失稳倒塌;

\quan{4} 由于施工方案与设计方案不符而导致的结构失衡而导致模板或支架倒塌;

\quan{5} 由于构建不齐全而导致的体系坍塌;

\quan{6} 由于模板超载使用,或堆放物过于集中而导致的模板破碎;

\quan{7} 由于违规在模板支架上悬挂起重设备、或是混凝土泵送管搭在模板支架而导致的局部荷载超过设计值而造成模板坍塌;

\quan{8} 高大跨模板搭设是未指定检测方案,或是布置检测点位不足而导致的支撑体系变形,
或是地基不均匀沉降。

\subsubsection{混凝土工程危险源辨识}

\quan{1} 由于运送混凝土时操作人重心失稳或脱把而造成的高处坠落;

\quan{2} 由于运送车道上有杂物而导致的运送推车失稳而造成的高处坠落或物体打击;

\quan{3} 由于运送车道过窄而造成的高处坠落或是物体打击;

\quan{4} 由于泵送混凝土管道架设不规范、不牢靠而导致的机械伤害;

\quan{5} 由于混凝土泵未放置稳妥就开始作业而造成的机械伤害;

\quan{6} 由于作业前未进行试压操作而造成的机械伤害或压力容器爆炸;

\quan{7} 由于乱搭乱接泵送管道或其他工具而造成的储电;

\quan{8} 由于操作工具无接地,无绝缘货物或无漏电保护而造成的触电;

\quan{9} 由于混凝土爆模而导致荷载过于集中而超过最大允许荷载而发生坍塌;

\quan{10} 由于临边洞口未经防护而造成的高处坠落或物体打击。

\subsubsection{脚手架工程危险源辨识}

\quan{1} 由于连墙件未按规定搭设、随意拆除、搭设位置或搭设结构不合理而造成的架体倾倒;

\quan{2} 由于基础发生严重破坏或地基不均匀沉降而造成的架体倾倒;

\quan{3} 由于脚手架设计承载力不足而造成超载而引发的架体倾倒;

\quan{4} 由于脚手架作业成没有铺设妥善脚手板,或者是架体边缘的空隙过大而引发的高处坠落;

\quan{5} 由于作业人员高空抛物而造成的物体打击或是高处坠落;

\quan{6} 脚手架上随意堆放物品而造成的物体打击;

\quan{7} 工人作业未有佩戴防护措施而造成的物体打击或高处坠落。

\subsubsection{吊装作业危险源辨识}

\quan{1} 起重机在运行中对人体造成的挤压或撞击;

\quan{2} 起重机吊钩超载断裂、吊运时钢丝绳从吊钩中滑出;

\quan{3} 吊运中重物坠落造成物体打击,重物从空中落到地面又反弹伤人;

\quan{4} 钢丝绳或麻绳断裂造成重物下落;使用应报废的钢丝绳,使用的吊具吊运超过额定起重量的重物等造成重物下落;

\quan{5} 汽车起重机作业场所地面不平整、支撑不稳定、配重不平衡、重物超过额定起重量而造成起重机倾覆;

\quan{6} 风力过大、违章作业造成起重机倾覆;

\quan{7} 机械传动部分未加防护,造成机械伤害;违章在卷扬机钢丝绳上面通过,运动中的钢丝绳将人挤伤或绊倒;

\quan{8} 载货升降机违章载人;

\quan{9} 人站在起重臂下等危险区域。

\subsubsection{其他工程危险源辨识}

\subsection{安全评价}
\subsubsection{评价依据}
\subsubsection{评价目的与评价范围}
\subsubsection{安全评价方法}
\subsubsection{评价单元的划分}
\subsubsection{基坑坍塌事故故障树法安全分析}
\subsubsection{模板工程坍塌事故故障树法安全分析}
\subsubsection{高处坠落事故故障树法安全分析}
\subsubsection{物料提升机与施工升降机安全检查表法安全分析}
\subsubsection{施工用电安全检查表法安全分析}
\subsubsection{脚手架工程预先危害分析法安全分析}

%\section{脚手架工程专项安全施工方案}
\subsection{编制依据}

\subsection{工程概况}

\subsection{脚手架搭设要求}
\subsubsection{落地式脚手架搭设要求}

\subsubsection{悬挑式脚手架搭设要求}

\subsection{方案选择}

\subsection{主要参数}

\subsection{脚手架计算书}
\subsubsection{双排脚手架安全性验算}

\subsection{脚手架质量验收}

\subsection{安全技术措施}
\subsubsection{脚手架搭设安全技术措施}

\subsubsection{脚手架拆除安全技术措施}

%\section{模板工程专项安全施工方案}
\subsection{编制依据}

(1) 《建筑工程施工质量验收统一标准》(GB50300-2001)

(2) 《建筑结构荷载规范》(GB50009-2001)

(3) 《建筑施工模板安全技术规范》(JGJ162-2008)

(4) 《建筑施工扣件式钢管脚手架安全技术规范》(JGJ130-2011)

(5) 《建筑施工脚手架安全技术统一标准》(GB51210-2016)

(6) 《建筑施工安全检查标准》(JGJ59-2011)

(7) 《混凝土结构工程施工及验收规范》(GB50204-2015) 

\subsection{模板支撑架搭设要求}

(1) 必须设置纵横扫地杆。纵向扫地杆应采用直角扣件固定在距底座上皮不大于200mm 处的立杆上,
横向扫地杆也应采用直角扣件固定在紧靠纵向扫地杆下方的立杆上。当立杆基础不在同一高度上时,必须将高出的纵向扫地杆向低处延长两跨与立杆固定,高低差不应大于 1m。

(2) 立杆应采用对接接头,且接头位置不应设置在同一步内,同一步立杆的两个相隔接头在高度方向错开的距离不宜小于 500,
各接头中心至柱节点的距离不宜大于步距的 1/3。

(3) 纵向水平杆接长宜采用对接扣件连接,对接扣件应交错布置,两根相邻纵向水平杆接头不宜设置在同步或同跨内,
不同步或不同跨两个相邻接头在水平方向错开的距离不应小于 500m,各接头中心至最近主节点的距离不宜大于纵距的 1/3。

(4) 搭接长度不应小于 1m,应等距离设置 3 个旋转扣件固定,端部扣件盖板边缘至搭接纵向水平杆端的距离不应小于 100mm。

\subsection{模板计算书}
\subsubsection{基本参数}

(1) 钢筋混凝土板厚 $120mm$,板的模板采用木胶合板厚 $15mm$,面板下次愣采用 $40\times 80mm$ 木方,间距 $300mm$,主楞采用 $60\times 100mm$ ,木方间距 $500mm$,支撑体系采用 $\phi 48.3\times 3.6mm$ 钢管。

(2) 梁尺寸 $300\times 700mm$,模板采用木胶合板厚 $18mm$,侧模次楞采用 $40\times 80mm$ 木方, 布置间距 $200mm$,采用 $M20$ 对拉螺栓加固,
间距为 $250mm$,梁底次楞采用 $40\times 80mm$ 木方,次楞间距 $150mm$,梁模板主楞采用 $40\times 80mm$ 木方,间距 $500mm$,支撑体系采用 $Φ48.3\times 3.6mm$ 钢管。

(3) 柱尺寸 $500\times 500mm$ ,计算高度取 $9.0m$,模板采用木胶合板厚 $18mm$,次楞采用 $40\times 80mm$ 木方,间距为 $115mm$,
柱箍间距 $400mm$,最下一层柱箍距柱根间距为 $200mm$,最上一层柱箍距柱顶间距为 $200mm$,柱箍螺栓采用 $M18$。


\subsubsection{模板安全性验算}

(1) 对板的安全性验算\\

\quan{1} 楼板验算\\

根据《建筑结构荷载规范》(GB50009-2012),查得相关构件的标准荷载值如下:

模板自重标准值取 $G_{1k}=0.5 kN/m^2$

混凝土自重标准值取 $G_{2k}=24\times 0.12=2.88 kN/m^2 $

钢筋自重标准值取 $G_{3k}=1.1\times 0.12=0.132 kN/m^2$

施工活荷载标准值取 $2.5kN$($2.5kN/m$)

当活荷载作为均布线性荷载控制时:

\[q_1=0.9\times 1.0\times [1.2(500+2880+132)+1.4\times 2500]=6942 N/m\]

当恒荷载作为均布线性荷载控制时:

\[q_1=0.9\times 1.0\times [1.35(500+2880+132)+0.7\times 1.4\times 2500]=6471 N/m\]

$q$ 取较大值,故 $q=6942 N/m$

当作为集中荷载控制时:

\begin{align*}
    q_2&=0.9\times 1.0\times 1.2\times (500+2880+132)=3792 N/m\\
    P&=0.9\times 1.0\times 1.4\times 2500=3150N
\end{align*}

施工荷载为均布线荷载时的弯矩值为:

\begin{align}
    \label{fx:5.0}
    M_1=0.1ql^2=0.1\times 6.942\times 0.32=0.06kN\cdot m
\end{align}

施工荷载为集中荷载时的弯矩值为:

\begin{align}
    M_2&=0.1ql^2+0.213pl\\
    &=0.1\times 3.79\times 0.32+0.213\times 3.15\times 0.3 \notag\\
    &=0.23 kN \cdot m \notag
\end{align}

弯矩取最大值计算,故 $M=0.23kN\cdot m$\\

对板做强度验算,按照三跨连续梁计算,

\begin{align}
    \label{fx:5.1}
    \sigma =\frac{M}{W}
\end{align}

式中,$M$ 为模板受到最大弯矩值;$W$ 为截面模量,矩形的截面模量公式为:

\begin{align}
    \label{fx:5.2}
    W=bh^2/6
\end{align}

式中,$b$ 为模板计算宽度;$h$ 为模板厚度。代入数据可得:

\begin{align*}
    W&=1000\times 152/6=37500 mm^3\\
    \sigma &=230000/37500=6.1 N/mm2<f=22 N/mm^2
\end{align*}

故板的强度满足要求。\\

对板做挠度验算,按照三跨连续梁计算,

\[q=0.9\times 1.0\times (500+2880+132)=3002 N/m\]

\begin{align}
    \label{fx:5.3}
    V_{max}=\frac{0.990ql^4}{100EI}
\end{align}

$E$ 为模板的弹性模量,取 $E=10000 N/mm^2$;$I$ 为截面惯性矩,计算公式为:

\begin{align}
    \label{fx:5.4}
    I=\frac{bh^3}{12}
\end{align}

代入数据得 $I=1000\times 15^3/12=281250 mm^4$。将截面惯性矩代入公式 \ref{fx:5.3} 得

\begin{align*}
    V&=\frac{0.99\times (3.002\times 10^2\times 300\times 10^2)}{100\times 10000\times 281250}\\
    &=0.058 mm < [v]=1/400=0.75 mm
\end{align*}

故板的挠度满足要求。\\

\quan{2} 板的次楞验算\\

板次楞采用 40×80mm 木方,间距 300mm,主楞间距 500mm。次楞的荷载按三跨连续梁计算受力,对板的次楞做强度验算:

\begin{align*}
    q_1&=0.9\times [1.2(500+2880+132)+1.4\times 2500]\times 0.3=1972 N/m \\
    q_2&=0.9\times (500+2880+132)\times 0.3=1.00 N/m
\end{align*}

分别按照公式 \ref{fx:5.0}、\ref{fx:5.2} 计算得出最大弯矩值与截面模量:

\begin{align*}
    M&=0.1\times 1.972\times 0.5^2=0.047 kN \cdot m\\
    W&=40\times 80^2 /6=42667 mm^2
\end{align*}

随后按照公式 \ref{fx:5.1} 求出板的次楞的强度:

\[
    \sigma = \frac{47000}{42667}=1.112 N/mm^2< f=17N/mm^2
\]

故次楞的强度满足要求。\\

对板的次楞做挠度验算,根据公式 \ref{fx:5.4} 算出次楞的最大截面惯性矩 $I$

\[
    I=40\times 80^3 /12=1706667 mm^4
\]

再将结果代回公式 \ref{fx:5.3},便可求出次楞的挠度设计值:

\begin{align*}
    V&=\frac{0.99\times 1 \times 500^4}{100\times 10000\times 1706667}\\
    &=0.0204 mm<[v]=1/400=1.25mm
\end{align*}

故次楞的挠度也满足要求。\\

\quan{3} 板的主楞验算\\

主楞采用 $60\times 100mm$ 木方,间距 $500mm$,板支撑在主楞上间距 $900mm$。由于木楞自重较小,因此计算时将其忽略
,主楞按照三跨梁计算,对主楞做强度验算:

\begin{align*}
    q_1&=0.9\times [1.2(500+2880+132)+1.4\times 2500]\times 0.5=3781 N/m \\
    q_2&=0.9\times (500+2880+132)\times 0.5=1.58 N/m
\end{align*}

按照主楞的最大弯矩公式求出主楞的最大弯矩值:

\begin{align}
    M=0.289q_1l=0.289\times 3.78\times 0.9=0.98 kN \cdot m
\end{align}

按照公式\ref{fx:5.2} 计算得出主楞的截面模量:

\[
    W=60\times 100^2 /6=100000 mm^2
\]

随后按照公式 \ref{fx:5.1} 求出板的主楞的强度:

\[
    \sigma = \frac{980000}{100000}=9.8 N/mm^2< f=17N/mm^2
\]

故主楞的强度满足要求。\\

对板的主楞做挠度验算,根据公式 \ref{fx:5.4} 算出次楞的最大截面惯性矩 $I$

\[
    I=60\times 100^3 /12=5000000 mm^4
\]

再将结果代回公式 \ref{fx:5.3},便可求出次楞的挠度设计值:

\begin{align*}
    V&=\frac{0.99\times 1580 \times 900^3}{100\times 10000\times 5000000}\\
    &=0.23 mm<[v]=1/400=2.25mm
\end{align*}

故主楞的挠度也满足要求。\\

\quan{4} 板的立杆验算\\

楼板支模高度 $9.0m$,属于高支模,立杆采用 $\phi 48.3\times 3.6$ 钢管,底部设置一道拉结杆,向上每 $1.5m$ 设置一道拉结杆。

查规范可得各个关键部件的自重标准值:

支撑自重为 $0.1444\times 9.0=1.29kN$

混凝土自重为 $24\times 0.12\times 0.5\times 0.9=1.29kN $

钢筋自重为 $1.1\times 0.12\times 0.5\times 0.9=0.05kN$

按照各个部件的自重标准值,能够求出恒荷载与活荷载的值:

恒荷载为 $1.27+1.29+0.05=2.63kN$

活荷载为 $2.5\times 0.5\times 0.9=1.125kN$

当活荷载控制时:

\[N_1=0.9\times (1.2\times 2.63+1.4\times 1.125)=4.25kN\]

当恒荷载控制时:

\[N_2=0.9\times (1.35\times 2.63+1.4\times 0.7\times 1.125=4.18kN\]

立杆轴向力取值取最大荷载值,即 $N=4.25 kN$

计算满堂脚手架的立杆长度的公式如下:

\begin{align}
\label{fx:5.5}
l_0&=k\mu_1(h+2a)\\
\label{fx:5.6}
l_0&=k\mu_2h
\end{align}

式中:
$k$ 为满堂支撑架立杆计算长度附加系数,取 1.185;   

$h$ 为步距 1.5m;                           

$a$ 为立杆伸出顶层水平杆中心线至支撑点长度 20mm;

$\mu_1$、$\mu_2$ 为满堂支撑架整体稳定因素的单杆计算长度系数,
根据《建筑施工扣件式脚手架安全技术规范》取值为 $\mu_1=1.54$、$\mu_2=1.951$

将上述数据代入公式 \ref{fx:5.5}、\ref{fx:5.6} 得:

\begin{align*}
    l_1&=1.185\times 1.54\times (1.5+2\times 0.2)=3.467 m\\
    l_2&=1.185\times 1.951\times 1.5=3.468 m
\end{align*}

$l_0$ 取二者最大值,即 $l_0=3.468 m$,随后根据公式 $\lambda=l_0/i$ 求出长细比为

\[
    \lambda = \frac{3.468}{1.59}=218
\]

查表可得轴心受压构件的稳定系数 $\phi =0.153$,再将上述所有数据代入下式可得出杆的弯曲正应力 $\sigma $

\begin{align}
\sigma &=\frac{N}{\phi A}\\
&=\frac{4180}{0.153\times 506} \notag\\
&=53.9 N/mm^2<f=205 N/mm^2 \notag
\end{align}

故主杆满足设计要求。\\

(2) 对梁的安全性验算\\



(3) 对柱的安全性验算\\



\subsection{模板安装及拆除}
\subsubsection{模板安装}

\subsubsection{模板拆除}

(1) 模板拆除应该经过相关管理人员的批准,按照国家标准《混凝土结构工程施工质量验收规范》(GB50204)的有关规定执行;

(2) 当混凝土未达规定强度,或已经达到规定强度,但需提前拆模的,必须经过计算和主管部门的批准后方可拆除;

(3) 大体积混凝土的拆模时间除了要满足混凝土强度要求之外,还应使得混凝土内外温差降低到 25 摄氏度以下时方可拆模;

(4) 拆模前应确定所使用的工具有效可靠,扳手等工具必须挂在工具袋内;

(5) 拆模的顺序和方法应该按照模板的设计规定执行,当无设计要求时,应按照先支的后拆、后支的先拆、先拆非承重模板
后拆承重模板,并应从上至下进行拆除;

(6) 多人同时作业时,应分工明确,统一信号,应预留出足够的操作面,人员要站在安全处;

(7) 在拆除互相搭连并影响其他后拆模板的支撑时,应布设临时支撑。拆模时逐块拆卸,不得使用大锤和撬棍成片撬落或砸倒拉倒;

(8) 拆除有洞口的模板时,应采取防止操作人员坠落的措施,洞口模板拆除后,应按国家先行标准《建筑施工高处作业安全技术规范》(JGJ80)
的有关规定进行防护。

\subsection{模板工程安全措施}

(1) 模板施工属高空作业,作业人员必须佩戴安全帽,安全绳并设置妥当,经医生检查认为不适宜高空作业的人员,不得进行高空作业

(2) 工作前应检查使用的工具是否牢固可用,扳手等工具必须用绳系在身上,钉子必须放在工具袋内;

(3) 安装与拆除五米以上的模板时,应搭设脚手架并设置防护栏杆,严谨上下共同作业;

(4) 遇六级以上的大风时应停止高空作业,雨雪后应先清扫施工现场,等待场地不滑时在进行作业;

(5) 两人抬运模板时要互相配合,协同工作。传递模板和工具时应先用绳子系牢固后再升降,不得随意乱抛。钢模板装拆时上下应有人接应,钢模板及其配件应随装随拆随运;

(6) 不得在脚手架上堆放模板;

(7) 支撑不得搭设在门窗框和脚手架上,斜撑和拉杆应设置在 1.8m 以上

(8) 支模过程中如遇中途停止,应将支撑,搭头,柱头等固定妥善,拆模间歇时应将已活动的模板和支撑等运走或妥善堆放,防止踏空;

(9) 模板上的预留洞口,应在留设后盖好;混凝土板上的预留洞口应在模板拆除之后盖好。

\subsection{成品保护}

(1) 对违反模板安全操作规范的,有损害模板安全使用现象的行为应及时制止个纠正,多层板拆除后应及时清理并分规格摆放,码放的高度不得大于 1.5 米;

(2) 吊装物体时,应轻吊轻放,不准碰撞已施工模板;

(3) 不经相关人员同意,任何人不得私自拆模,拆除模板以不损坏墙体,表面及棱角为准;

(4) 安装和拆除模板时不得用大锤砸多层板,以免使多层板翘曲变形以及砸坏硂成品;

(5) 多层板运输和堆放时,应做好防水工作,堆放场地也要有排水措施。


%=============  结论  ======================
\begin{center}
    \section*{\zihao{-2}  \textbf{结 ~~ 论}}
    \end{center}
\addcontentsline{toc}{section}{结论}

对于理工类专业来说,毕业设计是一个能够考验学生综合水平的一环,毕业设计可
以很好地培养学生的综合能力,并且将大学四年所学到的知识整合起来,可以提高对规
范的理解,对课程的认知,并且能有机的把理论和实际结合起来。除此之外毕业设计也
是一次能对专业软件和画图软件,以及排版能力的综合考量。

本设计是“河南省旅游中心项目安全施工组织设计”,通过本次的毕业设计,我能
对建筑施工的流程有了一个新的认识,对现场的安全技术和安全管理措施有了全新的理
解。建筑安全作为施工过程中看似平常的一环,实际上对项目的按时保质保量完工,人
员的生命安全,项目的经济效益等都有举足轻重的地位,可以说把控好安全的项目,无
论是施工质量还是整体成本,都会是非常优秀的一个项目。建筑作为国之基础,关乎着
社会的平稳运行和经济的发展,只有安全的做完施工,项目才可能顺利完成;只有项目
顺利完成,社会的经济水平才能得到提升,因此,安全管理在今后的现在施工中显得十
分重要。

在本设计中,首先考察了拟建项目的地理位置,分析了周边的环境影响,随后对工
程进行了危险源辨识,定性的对危险的物或因素进行了分析并提出了改进和防治措施,
然后针对重点的脚手架工程、模板工程和基坑工程定量做了专项的安全设计方案,最后
在保证安全的情况下尽可能做到文明施工。针对项目上可能出现的常见事故,还提出了
应急预案和演练过程,从而基本完备的达到了安全设计的要求。
在本次设计中,我可以说是真真正正了解到了安全工程专业日后的工作方向,将四
年所学习的课程串联了起来,真正的了解到了安全评价、土木施工、基础计算以及应急
管理等课程的知识要点,并将理论与实际结合了起来,我觉得这是一个工科大学生应该
有的技能和能力。在本次设计中,通过查阅资料,老师辅导还有同学讨论,我最终完成
了这个项目的安全组织设计,同时我也体会到了土建人的不易。本次毕业设计让我留下
了难忘的经历,我也相信我以后的工作和生活会做得更好。
%============= 参考文献 =====================
\addcontentsline{toc}{section}{参考文献}
{\zihao{5} \songti \bibliography{bibfile}}
\clearpage
\include{body/appendices}
%=============  致谢  ======================
\begin{center}
    \section*{\zihao{-2}  \textbf{致 ~~ 辞}}
    \end{center}
\addcontentsline{toc}{section}{致辞}

\end{document}
%%%%%%%%%% 结束 %%%%%%%%%%
